
% utf-8 ru, unix eolns
\documentclass[12pt,a4paper,oneside]{extarticle}
    \righthyphenmin=2 %минимально переносится 2 символа %%%
    \sloppy

% Рукопись оформлена в соответствии с правилами оформления 
% электронной версии авторского оригинала, 
% принятыми в Издательстве МГТУ им. Н.Э. Баумана.

\usepackage{geometry} % А4, примерно 28-31 строк(а) на странице 
    \geometry{paper=a4paper}
    \geometry{includehead=false} % Нет верх. колонтитула
    \geometry{includefoot=true}  % Есть номер страницы
    \geometry{bindingoffset=0mm} % Переплет    : 0  мм
    \geometry{top=20mm}          % Поле верхнее: 20 мм
    \geometry{bottom=25mm}       % Поле нижнее : 25 мм 
    \geometry{left=25mm}         % Поле левое  : 25 мм
    \geometry{right=25mm}        % Поле правое : 25 мм
    \geometry{headsep=10mm}  % От края до верх. колонтитула: 10 мм
    \geometry{footskip=20mm} % От края до нижн. колонтитула: 20 мм 

\usepackage{cmap}
\usepackage[T2A]{fontenc} 
\usepackage[utf8x]{inputenc}
\usepackage[english,russian]{babel}
\usepackage{misccorr}

\usepackage{amsmath}
\usepackage{amsfonts}
\usepackage{amssymb}

%\usepackage{cm-super} %человеческий рендер русских шрифтов

\setlength{\parindent}{1.25cm}  % Абзацный отступ: 1,25 см
\usepackage{indentfirst}        % 1-й абзац имеет отступ

\usepackage{setspace}   

\onehalfspacing % Полуторный интервал между строками

\makeatletter
\renewcommand{\@oddfoot }{\hfil\thepage\hfil} % Номер стр.
\renewcommand{\@evenfoot}{\hfil\thepage\hfil} % Номер стр.
\renewcommand{\@oddhead }{} % Нет верх. колонтитула
\renewcommand{\@evenhead}{} % Нет верх. колонтитула
\makeatother

\usepackage{fancyvrb}


\usepackage[pdftex]{graphicx}  % поддержка картинок для пдф
\graphicspath{ {./pictures/} }
\usepackage{rotating}
%\DeclareGraphicsExtensions{.jpg,.png}




\renewcommand{\labelenumi}{\theenumi.} %меняет вид нумерованного списка

\usepackage{perpage} %нумерация сносок 
\MakePerPage{footnote}

\usepackage[all]{xy} %поддержка графов

\usepackage{listings} %листинги
\renewcommand{\lstlistingname}{Листинг}
\lstset{
  basicstyle=\tiny,
  breaklines=true
  }


\usepackage{url}


\usepackage{tikz} %для рисования графиков
\usepackage{pgfplots}

\usepackage{gensymb}

\usepackage{ccaption}%изменяет подпись к рисунку
\makeatletter 
\renewcommand{\fnum@figure}[1]{Рисунок~\thefigure~---~\sffamily}
\makeatother

\begin{document}
\pgfplotsset{compat=1.8}

\thispagestyle{empty}
\newpage
{
\centering


\textbf{
МОСКОВСКИЙ ГОСУДАРСТВЕННЫЙ ТЕХНИЧЕСКИЙ УНИВЕРСИТЕТ ИМЕНИ Н. Э. БАУМАНА \\
Факультет информатики и систем управления \\
Кафедра теоретической информатики и компьютерных технологий}
\bigskip
\bigskip
\bigskip
\bigskip
\bigskip
\bigskip
\bigskip

\vfill


Лабораторная работа №4 \\
по курсу <<Теория игр и исследование операций>>

\bigskip

{\large <<Матричные игры с нулевой суммой. Смешанные стратегии>>}
\bigskip

\vfill



\hfill\parbox{4cm} {
Выполнил:\\
студент ИУ9-111 \hfill \\
Выборнов А.И.\hfill \medskip\\
Руководитель:\\
Басараб М.А.\hfill
}


\vspace{\fill}

Москва \number\year
\clearpage
}



\clearpage
% Отчет должен содержать: титульный лист; цель работы; постановку задачи; решение
% матричной игры в смешанных стратегиях симплекс-методом за обоих игроков (прямая
% и двойственная задачи ЛП).

\section{Цель работы}
    Постановка антагонистической игры двух лиц в нормальной форме. Найти решение игры в смешанных стратегиях (стратегическую седловую точку) за обоих игроков.

\section{Постановка задачи}
    Матрица стратегий имеет вид:

    \begin{center}
        \begin{tabular}{|c|c|c|c|c|}
            \hline
            Стратегии & $b_1$ & $b_2$ & $b_3$ & $b_4$ \\ \hline
            $a_1$     & 4     & 1     & 17    & 18    \\ \hline
            $a_2$     & 4     & 14    & 6     & 16    \\ \hline
            $a_3$     & 0     & 14    & 14    & 13    \\ \hline
            $a_4$     & 6     & 13    & 4     & 15    \\ \hline
            $a_4$     & 12    & 11    & 3     & 16    \\ \hline
        \end{tabular}
    \end{center}

    Необходимо решить матричную игру в смешанных стратегиях за обоих игроков.

\section{Решение}
    Найдем смешанные стратегии для игрока А. Для этого составим систему уравнений:
    \begin{gather}
        \begin{cases}
            4x_1 + 4x_2 + 6x_4 + 12x_5 \geq g, \\
            x_1 + 14x_2 + 14x_3 + 13x_4 + 11x_5 \geq g, \\
            17x_1 + 6x_2 + 14x_3 + 4x_4 + 3x_5 \geq g, \\
            18x_1 + 16x_2 + 13x_3 + 15x_4 + 16x_5 \geq g, \\
            x_1+x_2+x_3+x_4+x_5=1.
        \end{cases} \notag
    \end{gather} 
    где g --- минимальный выигрыш.

    Разделим систему на функцию g:
    \begin{gather}
        \begin{cases}
            4u_1 + 4u_2 + 6u_4 + 12u_5 \geq 1, \\
            u_1 + 14u_2 + 14u_3 + 13u_4 + 11u_5 \geq 1, \\
            17u_1 + 6u_2 + 14u_3 + 4u_4 + 3u_5 \geq 1, \\
            18u_1 + 16u_2 + 13u_3 + 15u_4 + 16u_5 \geq 1, \\
            u_1+u_2+u_3=1/g.
        \end{cases} \notag
    \end{gather} 

    Сформулируем задачу для решения симплекс-методом:

    \begin{gather}
        W = u_1+u_2+u_3+u_4+u_5 \rightarrow min, \notag \\
        \begin{cases}
            4u_1 + 4u_2 + 6u_4 + 12u_5 \geq 1, \\
            u_1 + 14u_2 + 14u_3 + 13u_4 + 11u_5 \geq 1, \\
            17u_1 + 6u_2 + 14u_3 + 4u_4 + 3u_5 \geq 1, \\
            18u_1 + 16u_2 + 13u_3 + 15u_4 + 16u_5 \geq 1, \\
        \end{cases} \notag \\
        u_i \geq 0, i=1,2,3,4,5.\notag
    \end{gather}

    % Решаем: ...

    Находим оптимальное решение:
    \begin{gather}
        u_1 = \frac{1}{28}, u_2 = 0, u_3 = \frac{5}{392}, u_4 = 0, u_5 = \frac{1}{14}, \notag \\
        W = \frac{47}{392}, \notag \\
        g = \frac{1}{W} = \frac{392}{47}. \notag
    \end{gather}

    Оптимальные стратегии:
    \begin{gather}
        x_1 = u_1g = \frac{1}{28} * \frac{392}{47} = \frac{14}{47}, \notag \\
        x_2 = u_2g = 0, \notag \\
        x_3 = u_3g =\frac{5}{392} * \frac{392}{47} = \frac{5}{47}, \notag \\
        x_4 = u_4g = 0, \notag \\
        x_5 = u_5g = \frac{1}{14} * \frac{392}{47} =\frac{28}{47}, \notag
    \end{gather}

    Таким образом, оптимальная смешанная стратегия игрока A равна:
    \begin{gather}
        (\frac{14}{47}, 0, \frac{5}{47}, 0, \frac{28}{47}) \notag
    \end{gather}

    
    Для нахождения смешанной стратегии игрока B составим систему:
    \begin{gather}
        \begin{cases}
            4v_1 + v_2 + 17v_3 + 18v_4 \leq h, \notag \\
            4v_1 + 14v_2 + 6v_3 + 16v_4 \leq h, \notag \\
            14v_2 + 14v_3 + 13v_4 \leq h, \notag \\
            6v_1 + 13v_2 + 4v_3 + 15v_4  \leq h, \notag \\
            12v_1 + 11v_2 + 3v_3 + 16v_4  \leq h, \notag \\
            v_1+v_2+v_3+v_4=1. \notag
        \end{cases} \notag
    \end{gather} 
    где h – максимальный проигрыш игрока B.

    Разделим систему на h:
    \begin{gather}
        \begin{cases}
            4v_1 + v_2 + 17v_3 + 18v_4 \leq 1, \notag \\
            4v_1 + 14v_2 + 6v_3 + 16v_4 \leq 1, \notag \\
            14v_2 + 14v_3 + 13v_4 \leq 1, \notag \\
            6v_1 + 13v_2 + 4v_3 + 15v_4  \leq 1, \notag \\
            12v_1 + 11v_2 + 3v_3 + 16v_4  \leq 1, \notag \\
            v_1+v_2+v_3+v_4=1/h.
        \end{cases} \notag
    \end{gather}

    Сформулируем задачу для решения симплекс-методом:
    \begin{gather}
        v_1+v_2+v_3+v_4 \rightarrow max. \notag \\
        \begin{cases}
            4v_1 + v_2 + 17v_3 + 18v_4 \leq 1, \notag \\
            4v_1 + 14v_2 + 6v_3 + 16v_4 \leq 1, \notag \\
            14v_2 + 14v_3 + 13v_4 \leq 1, \notag \\
            6v_1 + 13v_2 + 4v_3 + 15v_4  \leq 1, \notag \\
            12v_1 + 11v_2 + 3v_3 + 16v_4  \leq 1, \notag \\
        \end{cases} \notag \\
        v_i \geq 0, i=1,2,3,4.\notag
    \end{gather} 

    Решение имеет вид:
    \begin{gather}
        v_1 = \frac{5}{196}, v_2 = \frac{6}{196}, v_3 = \frac{19}{392}, v_4= 0 \notag \\
        Z = \frac{47}{392}, \notag \\
        H = \frac{1}{W} = \frac{392}{47}. \notag
    \end{gather}

        Частоты выбора стратегий:
    \begin{gather}
        y_1 = v_1h = \frac{5}{196} * \frac{392}{47} = \frac{10}{47}, \notag \\
        y_2 = v_2h = \frac{6}{196} * \frac{392}{47} = \frac{12}{47}, \notag \\
        y_3 = v_3h = \frac{19}{392} * \frac{392}{47} = \frac{19}{47}, \notag \\
        y_4 = v_4h = 0, \notag 
    \end{gather}

    Таким образом, оптимальная смешанная стратегия игрока В равна: 
    $$(\frac{10}{47}, \frac{12}{47}, \frac{19}{47}, 0)$$

\end{document}