
% utf-8 ru, unix eolns
\documentclass[12pt,a4paper,oneside]{extarticle}
    \righthyphenmin=2 %минимально переносится 2 символа %%%
    \sloppy

% Рукопись оформлена в соответствии с правилами оформления 
% электронной версии авторского оригинала, 
% принятыми в Издательстве МГТУ им. Н.Э. Баумана.

\usepackage{geometry} % А4, примерно 28-31 строк(а) на странице 
    \geometry{paper=a4paper}
    \geometry{includehead=false} % Нет верх. колонтитула
    \geometry{includefoot=true}  % Есть номер страницы
    \geometry{bindingoffset=0mm} % Переплет    : 0  мм
    \geometry{top=20mm}          % Поле верхнее: 20 мм
    \geometry{bottom=25mm}       % Поле нижнее : 25 мм 
    \geometry{left=25mm}         % Поле левое  : 25 мм
    \geometry{right=25mm}        % Поле правое : 25 мм
    \geometry{headsep=10mm}  % От края до верх. колонтитула: 10 мм
    \geometry{footskip=20mm} % От края до нижн. колонтитула: 20 мм 

\usepackage{cmap}
\usepackage[T2A]{fontenc} 
\usepackage[utf8x]{inputenc}
\usepackage[english,russian]{babel}
\usepackage{misccorr}

\usepackage{amsmath}
\usepackage{amsfonts}
\usepackage{amssymb}

%\usepackage{cm-super} %человеческий рендер русских шрифтов

\setlength{\parindent}{1.25cm}  % Абзацный отступ: 1,25 см
\usepackage{indentfirst}        % 1-й абзац имеет отступ

\usepackage{setspace}   

\onehalfspacing % Полуторный интервал между строками

\makeatletter
\renewcommand{\@oddfoot }{\hfil\thepage\hfil} % Номер стр.
\renewcommand{\@evenfoot}{\hfil\thepage\hfil} % Номер стр.
\renewcommand{\@oddhead }{} % Нет верх. колонтитула
\renewcommand{\@evenhead}{} % Нет верх. колонтитула
\makeatother

\usepackage{fancyvrb}


\usepackage[pdftex]{graphicx}  % поддержка картинок для пдф
\graphicspath{ {./pictures/} }
\usepackage{rotating}
%\DeclareGraphicsExtensions{.jpg,.png}




\renewcommand{\labelenumi}{\theenumi.} %меняет вид нумерованного списка

\usepackage{perpage} %нумерация сносок 
\MakePerPage{footnote}

\usepackage[all]{xy} %поддержка графов

\usepackage{listings} %листинги
\renewcommand{\lstlistingname}{Листинг}
\lstset{
  basicstyle=\tiny,
  breaklines=true
  }


\usepackage{url}


\usepackage{tikz} %для рисования графиков
\usepackage{pgfplots}

\usepackage{gensymb}

\usepackage{ccaption}%изменяет подпись к рисунку
\makeatletter 
\renewcommand{\fnum@figure}[1]{Рисунок~\thefigure~---~\sffamily}
\makeatother

\begin{document}
\pgfplotsset{compat=1.8}

\thispagestyle{empty}
\newpage
{
\centering


\textbf{
МОСКОВСКИЙ ГОСУДАРСТВЕННЫЙ ТЕХНИЧЕСКИЙ УНИВЕРСИТЕТ ИМЕНИ Н. Э. БАУМАНА \\
Факультет информатики и систем управления \\
Кафедра теоретической информатики и компьютерных технологий}
\bigskip
\bigskip
\bigskip
\bigskip
\bigskip
\bigskip
\bigskip

\vfill


Лабораторная работа №5 \\
по курсу <<Теория игр и исследование операций>>

\bigskip

{\large <<Игры с природой. Критерии принятия решений>>}
\bigskip

\vfill



\hfill\parbox{4cm} {
Выполнил:\\
студент ИУ9-111 \hfill \\
Выборнов А.И.\hfill \medskip\\
Руководитель:\\
Басараб М.А.\hfill
}


\vspace{\fill}

Москва \number\year
\clearpage
}



\clearpage
% Отчет должен содержать: титульный лист; цель работы; постановку задачи; нахожде-
% ние оптимальной стратегии в соответствии с критериями Вальда, максимума, Гурвица
% и Сэвиджа; выбор рекомендуемой стратегии по принципу простого большинства «по-
% бед».

\section{Цель работы}
    Умение применять различные критерии (Бернулли, Вальда, максимума, Гурвица,
Сэвиджа) для выбора стратегии в условиях полной неопределенности.

\section{Постановка задачи}
    Матрица стратегий имеет вид:

    \begin{center}
        \begin{tabular}{|c|c|c|c|c|}
            \hline
            Стратегии & $b_1$ & $b_2$ & $b_3$ & $b_4$ \\ \hline
            $a_1$     & 4     & 1     & 17    & 18    \\ \hline 
            $a_2$     & 4     & 14    & 6     & 16    \\ \hline 
            $a_3$     & 0     & 14    & 14    & 13    \\ \hline 
            $a_4$     & 6     & 13    & 4     & 15    \\ \hline 
            $a_5$     & 12    & 11    & 3     & 16    \\ \hline 
        \end{tabular}
    \end{center}

    Необходимо найти стратегии игрока при реализации гипотез недостаточного основания (Бернулли), пессимизма (Вальда), оптимизма, смешанной (Гурвица) при $\alpha=0.5$, рисков (Сэвиджа) и рекомендовать выбор стратегии согласно принципу большинства.

\section{Решение}
    Если воспользоваться критерием Бернулли, то следует руководствоваться стратегией $a_5$. Соответствующее математическое ожидание выигрыша при этом максимально и равно 10.5.

    Пессимистическая стратегия (критерий Вальда) определяет выбор $a_2$ или $a_4$ (нижняя цена игры равна 4).

    Оптимистическая стратегия соответствует выбору $a_1$ (максимально возможный выигрыш 18).

    Критерий Гурвица определим из условия равновероятной реализации пессимистической и оптимистической гипотез ($\alpha = 0.5$). Наилучшая стратегия: $a_4$ (ожидаемый выигрыш равен 10).

    Cоставим теперь для рассматриваемой игры таблицу рисков:
    \begin{center}
        \begin{tabular}{|c|c|c|c|c|}
            \hline
            Стратегии & $b_1$ & $b_2$ & $b_3$ & $b_4$ \\ \hline
            $a_1$     & 8     & 13     & 0    & 0     \\ \hline 
            $a_2$     & 8     & 0      & 11   & 2     \\ \hline 
            $a_3$     & 12    & 0      & 3    & 5     \\ \hline 
            $a_4$     & 6     & 1      & 13   & 3     \\ \hline 
            $a_5$     & 0     & 3      & 14   & 2     \\ \hline 
        \end{tabular}
    \end{center}

    Максимальный риск минимален для стратегии $a_2$.

    Окончательно, согласно принципу большинства, следует рекомендовать выбор стратегии $a_2$ и $a_4$ --- лучшие по двум из пяти рассмотренных критериев. Следующие по значимости стратегии: $a_1$ и $a_5$ (лучшие по двум из пяти критериев).
\end{document}