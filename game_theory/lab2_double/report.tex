
% utf-8 ru, unix eolns
\documentclass[12pt,a4paper,oneside]{extarticle}
    \righthyphenmin=2 %минимально переносится 2 символа %%%
    \sloppy

% Рукопись оформлена в соответствии с правилами оформления 
% электронной версии авторского оригинала, 
% принятыми в Издательстве МГТУ им. Н.Э. Баумана.

\usepackage{geometry} % А4, примерно 28-31 строк(а) на странице 
    \geometry{paper=a4paper}
    \geometry{includehead=false} % Нет верх. колонтитула
    \geometry{includefoot=true}  % Есть номер страницы
    \geometry{bindingoffset=0mm} % Переплет    : 0  мм
    \geometry{top=20mm}          % Поле верхнее: 20 мм
    \geometry{bottom=25mm}       % Поле нижнее : 25 мм 
    \geometry{left=25mm}         % Поле левое  : 25 мм
    \geometry{right=25mm}        % Поле правое : 25 мм
    \geometry{headsep=10mm}  % От края до верх. колонтитула: 10 мм
    \geometry{footskip=20mm} % От края до нижн. колонтитула: 20 мм 

\usepackage{cmap}
\usepackage[T2A]{fontenc} 
\usepackage[utf8x]{inputenc}
\usepackage[english,russian]{babel}
\usepackage{misccorr}

\usepackage{amsmath}
\usepackage{amsfonts}
\usepackage{amssymb}

%\usepackage{cm-super} %человеческий рендер русских шрифтов

\setlength{\parindent}{1.25cm}  % Абзацный отступ: 1,25 см
\usepackage{indentfirst}        % 1-й абзац имеет отступ

\usepackage{setspace}   

\onehalfspacing % Полуторный интервал между строками

\makeatletter
\renewcommand{\@oddfoot }{\hfil\thepage\hfil} % Номер стр.
\renewcommand{\@evenfoot}{\hfil\thepage\hfil} % Номер стр.
\renewcommand{\@oddhead }{} % Нет верх. колонтитула
\renewcommand{\@evenhead}{} % Нет верх. колонтитула
\makeatother

\usepackage{fancyvrb}


\usepackage[pdftex]{graphicx}  % поддержка картинок для пдф
\graphicspath{ {./pictures/} }
\usepackage{rotating}
%\DeclareGraphicsExtensions{.jpg,.png}




\renewcommand{\labelenumi}{\theenumi.} %меняет вид нумерованного списка

\usepackage{perpage} %нумерация сносок 
\MakePerPage{footnote}

\usepackage[all]{xy} %поддержка графов

\usepackage{listings} %листинги
\renewcommand{\lstlistingname}{Листинг}
\lstset{
  basicstyle=\tiny,
  breaklines=true
  }


\usepackage{url}


\usepackage{tikz} %для рисования графиков
\usepackage{pgfplots}

\usepackage{gensymb}

\usepackage{ccaption}%изменяет подпись к рисунку
\makeatletter 
\renewcommand{\fnum@figure}[1]{Рисунок~\thefigure~---~\sffamily}
\makeatother

\begin{document}
\pgfplotsset{compat=1.8}

\thispagestyle{empty}
\newpage
{
\centering


\textbf{
МОСКОВСКИЙ ГОСУДАРСТВЕННЫЙ ТЕХНИЧЕСКИЙ УНИВЕРСИТЕТ ИМЕНИ Н. Э. БАУМАНА \\
Факультет информатики и систем управления \\
Кафедра теоретической информатики и компьютерных технологий}
\bigskip
\bigskip
\bigskip
\bigskip
\bigskip
\bigskip
\bigskip

\vfill


Лабораторная работа №2 \\
по курсу <<Теория игр и исследование операций>>

\bigskip

{\large <<Двойственность в линейном программировании>>}
\bigskip

\vfill



\hfill\parbox{4cm} {
Выполнил:\\
студент ИУ9-111 \hfill \\
Выборнов А.И.\hfill \medskip\\
Руководитель:\\
Басараб М.А.\hfill
}


\vspace{\fill}

Москва \number\year
\clearpage
}



\clearpage

% Отчет должен содержать: титульный лист; цель работы; постановку задачи; решение
% исходной ПЗ ЛП симплекс-методом; запись ДЗ ЛП; запись ДЗ в канонической форме;
% исходную симплекс-таблицу; оптимальное решение; сравнение решений ПЗ и ДЗ.

\section{Цель работы}
    Постановка двойственной задачи (ДЗ) линейного программирования по прямой задаче (ПЗ). Решение соответствующей двойственной задачи по прямой задаче.

\section{Постановка задачи}
    Поставить и решить двойственную задачу, соответствующую приведённой прямой задаче:
    
    $$F = cx \rightarrow max,$$
    $$Ax \leq b,$$ 
    $$x = [x_1, x_2, x_3]^T,$$
    $$x_1, x_2, x_3 \geq 0.$$

    $$c = [2, 5, 3], 
    A = \begin{pmatrix}
        2 & 1 & 2\\
        1 & 2 & 0\\
        0 & 0.5 & 1\\
    \end{pmatrix}, 
    b^T = [6, 6, 2]$$


\section{Решение}
    Прямая задача:

    \begin{gather}
        F = 2x_1+5x_2+3x_3 \rightarrow max,  \notag  \\
        \begin{cases}
            2x_1 + x_2 + 2x_3 \leq 6, \\
            x_1 + 2x_2 \leq 6, \\
            0.5x_2 + x_3 \leq 2.
        \end{cases} \notag \\
        x_1, x_2, x_3 \geq 0. \notag
    \end{gather} 

    Находим решение ПЗ симплекс-методом (приведено в главе~\ref{sec:pz}):
    \begin{gather}
        \begin{cases}
            x_1 = 1,\\
            x_2 = 2.5, \\
            x_3 = 0.75. \\
        \end{cases} \notag \\
        max (F(x)) = 16.75 \notag 
    \end{gather}

    Согласно указанным правилам формулируем ДЗ ЛП:

    \begin{gather}
        \Phi = 6y_1 + 6y_2 + 2y_3 \rightarrow min,  \notag  \\
        \begin{cases}
            2y_1 + y_2 \geq 2, \\
            y_1 + 2y_2 + 0.5y_3 \geq 5, \\
            2y_1 + y_3 \geq 3.
        \end{cases} \notag \\
        y_1, y_2, y_3 \geq 0. \notag
    \end{gather} 

    Каноническая форма ДЗ имеет вид:
    \begin{gather}
        \Phi = 6y_1 + 6y_2 + 2y_3 \rightarrow min,  \notag  \\
        \begin{cases}
            2y_1 + y_2 - y_4 = 2, \\
            y_1 + 2y_2 + 0.5y_3 - y_5 = 5, \\
            2y_1 + y_3 - y_6 = 3.
        \end{cases} \notag \\
        y_1, y_2, y_3 \geq 0. \notag
    \end{gather} 

    Исходная симплекс-таблица записывается в виде:
    \begin{center}
        \begin{tabular}{|c|c|c|c|c|}
            \hline
                 & $s_{i0}$ & $y_1$ & $y_2$ & $y_3$ \\ \hline
            $y_4$ & -2      & -2    & -1    & 0 \\ \hline
            $y_5$ & -5      & -1    & -2    & -0.5 \\ \hline
            $y_6$ & -3      & -2    & 0     & -1 \\ \hline
            $F$   & 0       & 6     & 6     & 2 \\ \hline
        \end{tabular}
    \end{center}

    Симплекс таблица, задающая опорное решение:
    \begin{center}
        \begin{tabular}{|c|c|c|c|c|}
            \hline
                 & $s_{i0}$ & $y_1$ & $y_4$ & $y_6$ \\ \hline
            $y_2$ & 2       & 2     & -1    & 0 \\ \hline
            $y_3$ & 3       & 2     & 0     & -1 \\ \hline
            $y_5$ & 0.5     & 4     & -2    & -0.5 \\ \hline
            $F$   & -18     & -10   & 6     & 2 \\ \hline
        \end{tabular}
    \end{center}
    
    
    Таблица, определяющая оптимальный план решения задачи:
    \begin{center}
        \begin{tabular}{|c|c|c|c|c|}
            \hline
                 & $s_{i0}$ & $y_4$ & $y_5$ & $y_6$ \\ \hline
            $y_2$ & 1.75    & 0     & -0.5  & 0.25 \\ \hline
            $y_3$ & 2.75    & 1     & -0.5  & -0.75 \\ \hline
            $y_1$ & 0.13    & -0.5  & 0.25  & -0.13 \\ \hline
            $F$   & -16.75  & 1     & 2.5   & 0.75 \\ \hline
        \end{tabular}
    \end{center}

    Решение ДЗ:

    \begin{gather}
        \begin{cases}
            y_1 = 0.13,\\
            y_2 = 1.75, \\
            y_3 = 2.75. \\
        \end{cases} \notag \\
        max (\Phi(x)) = 16.75 \notag 
    \end{gather}

    Таким образом, решения прямой и двойственной задач совпадают.

\clearpage

\section{Решение ПЗ}
    \label{sec:pz}
    \begin{gather}
        F = 2x_1+5x_2+3x_3 \rightarrow max,  \notag  \\
        \begin{cases}
            2x_1 + x_2 + 2x_3 \leq 6, \\
            x_1 + 2x_2 \leq 6, \\
            0.5x_2 + x_3 \leq 2.
        \end{cases} \notag \\
        x_1, x_2, x_3 \geq 0. \notag
    \end{gather} 

    Избавимся от неравенства - получим задачу в канонической форме:
    \begin{gather}
        F = 2x_1+5x_2+3x_3 \rightarrow max,  \notag  \\
        \begin{cases}
            2x_1 + x_2 + 2x_3 + x_4 = 6, \\
            x_1 + 2x_2 + x_5 = 6, \\
            0.5x_2 + x_3 + x_6 = 2.
        \end{cases} 
        \notag \\ x_1, x_2, x_3, x_4, x_5 \geq 0. \
    \end{gather} 

    Пусть $x_4,x_5,x_6$ --- базисные переменные, $x_1,x_2,x_3$ --- свободные переменные. Тогда имеем:
    \begin{gather}
        F = 2x_1+5x_2+3x_3 \rightarrow max,  \notag  \\
        \begin{cases}
            x_4 = 6 - (2x_1 + x_2 + 2x_3), \\
            x_5 = 6 - (x_1 + 2x_2),\\
            x_6 = 2 - (0.5x_2 + x_3).
        \end{cases} 
        \notag \\ x_1, x_2, x_3, x_4, x_5 \geq 0. \
    \end{gather} 

    Исходная симплекс-таблица записывается в виде:
    \begin{center}
        \begin{tabular}{|c|c|c|c|c|}
            \hline
                 & $s_{i0}$ & $x_1$ & $x_2$ & $x_3$ \\ \hline
            $x_4$ & 6       & 2     & 1     & 2 \\ \hline
            $x_5$ & 6       & 1     &{\bf 2}& 0 \\ \hline
            $x_6$ & 2       & 0     & 0.5   & 1 \\ \hline
            $F$   & 0       & -2    & -5    & -3 \\ \hline
        \end{tabular}
    \end{center}

    Так как в столбце свободных членов нет отрицательных элементов, то найдено опорное решение: $x=[0, 0, 0, 6, 6, 2], F(x)=0$. В строке F имеются отрицательные элементы, это означает что полученое решение не оптимально.

    $x_2$ --- разрешающий столбец, так как значение в строке таблицы, соответствующей целевой функции по модулю максимально.

    Найдем минимальное положительное отношение элемента свободных членов $s_{i0}$ к cоответствующем элементу в разрешающем столбце. Минимальное положительное отношение в строке $x_5$, выберем её в качестве разрешающей.

    Пересчитываем симплекс таблицу:
    \begin{center}
        \begin{tabular}{|c|c|c|c|c|}
            \hline
                 & $s_{i0}$ & $x_1$ & $x_5$ & $x_3$ \\ \hline
            $x_4$ & 3       & 1.5   & -0.5  & 2  \\ \hline
            $x_2$ & 3       & 0.5   & 0.5   & 0  \\ \hline
            $x_6$ & 0.5     & -0.25 & -0.25 &{\bf 1}\\ \hline
            $F$   & 15      & 0.5   & 2.5   & -3 \\ \hline
        \end{tabular}
    \end{center}

    В строке F имеются отрицательные элементы, это означает что полученое решение не оптимально.
    В качестве разрешающего столбца выбираем $x_3$ и в качестве разрешающей строки выбираем $x_6$ (причины выбора аналогичны описанным выше).

    Пересчитываем симплекс таблицу:
    \begin{center}
        \begin{tabular}{|c|c|c|c|c|}
            \hline
                 & $s_{i0}$ & $x_1$ & $x_5$ & $x_6$ \\ \hline
            $x_4$ & 2       &{\bf 2} & 0     & -2 \\ \hline
            $x_2$ & 3       & 0.5   & 0.5   & 0  \\ \hline
            $x_3$ & 0.5     & -0.25 & -0.25 & 1 \\ \hline
            $F$   & 16.5    & -0.25 & 1.75  & 3  \\ \hline
        \end{tabular}
    \end{center}

    В строке F имеются отрицательные элементы, это означает что полученое решение не оптимально.
    В качестве разрешающего столбца выбираем $x_1$ и в качестве разрешающей строки выбираем $x_4$ (причины выбора аналогичны описанным выше).

    Пересчитываем симплекс таблицу:
    \begin{center}
        \begin{tabular}{|c|c|c|c|c|}
            \hline
                 & $s_{i0}$ & $x_4$   & $x_5$   & $x_6$ \\ \hline
            $x_1$ & 1       & 0.5     & 0       & 0.5     \\ \hline
            $x_2$ & 2.5     & -0.25   & 0.5     & 0     \\ \hline
            $x_3$ & 0.75    & 0.125   & -0.25   & 0.75  \\ \hline
            $F$   & 16.75   & 0.125   & 1.75    & 2.75  \\ \hline
        \end{tabular}
    \end{center}

    Среди значений индексной строки нет отрицательных. Поэтому таблица определяет оптимальное решение:   
    \begin{gather}
        \begin{cases}
            x_1 = 1,\\
            x_2 = 2.5, \\
            x_3 = 0.75. \\
        \end{cases} \notag \\
        max (F(x)) = 16.75 \notag 
    \end{gather}

    Проверим полученное решение на допустимость:
    \begin{gather}
        \begin{cases}
            x_1 = 1,\\
            x_2 = 2.5, \\
            x_3 = 0.75. \\
            x_4 = 6 - (2x_1 + x_2 + 2x_3) = 6 - (4.5 + 1.5)=0, \\
            x_5 = 6 - (x_1 + 2x_2) = 6 - (1+5) = 0,\\
            x_6 = 2 - (0.5x_2 + x_3) = 2 - (1.25+0.75)= 0.
        \end{cases} 
    \end{gather}
    Решение допустимое, так как все переменные неотрицательны.
\end{document}