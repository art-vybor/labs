
% utf-8 ru, unix eolns
\documentclass[12pt,a4paper,oneside]{extarticle}
    \righthyphenmin=2 %минимально переносится 2 символа %%%
    \sloppy

% Рукопись оформлена в соответствии с правилами оформления 
% электронной версии авторского оригинала, 
% принятыми в Издательстве МГТУ им. Н.Э. Баумана.

\usepackage{geometry} % А4, примерно 28-31 строк(а) на странице 
    \geometry{paper=a4paper}
    \geometry{includehead=false} % Нет верх. колонтитула
    \geometry{includefoot=true}  % Есть номер страницы
    \geometry{bindingoffset=0mm} % Переплет    : 0  мм
    \geometry{top=20mm}          % Поле верхнее: 20 мм
    \geometry{bottom=25mm}       % Поле нижнее : 25 мм 
    \geometry{left=25mm}         % Поле левое  : 25 мм
    \geometry{right=25mm}        % Поле правое : 25 мм
    \geometry{headsep=10mm}  % От края до верх. колонтитула: 10 мм
    \geometry{footskip=20mm} % От края до нижн. колонтитула: 20 мм 

\usepackage{cmap}
\usepackage[T2A]{fontenc} 
\usepackage[utf8x]{inputenc}
\usepackage[english,russian]{babel}
\usepackage{misccorr}

\usepackage{amsmath}
\usepackage{amsfonts}
\usepackage{amssymb}

%\usepackage{cm-super} %человеческий рендер русских шрифтов

\setlength{\parindent}{1.25cm}  % Абзацный отступ: 1,25 см
\usepackage{indentfirst}        % 1-й абзац имеет отступ

\usepackage{setspace}   

\onehalfspacing % Полуторный интервал между строками

\makeatletter
\renewcommand{\@oddfoot }{\hfil\thepage\hfil} % Номер стр.
\renewcommand{\@evenfoot}{\hfil\thepage\hfil} % Номер стр.
\renewcommand{\@oddhead }{} % Нет верх. колонтитула
\renewcommand{\@evenhead}{} % Нет верх. колонтитула
\makeatother

\usepackage{fancyvrb}


\usepackage[pdftex]{graphicx}  % поддержка картинок для пдф
\graphicspath{ {./pictures/} }
\usepackage{rotating}
%\DeclareGraphicsExtensions{.jpg,.png}




\renewcommand{\labelenumi}{\theenumi.} %меняет вид нумерованного списка

\usepackage{perpage} %нумерация сносок 
\MakePerPage{footnote}

\usepackage[all]{xy} %поддержка графов

\usepackage{listings} %листинги
\renewcommand{\lstlistingname}{Листинг}
\lstset{
  basicstyle=\tiny,
  breaklines=true
  }


\usepackage{url}


\usepackage{tikz} %для рисования графиков
\usepackage{pgfplots}

\usepackage{gensymb}

\usepackage{ccaption}%изменяет подпись к рисунку
\makeatletter 
\renewcommand{\fnum@figure}[1]{Рисунок~\thefigure~---~\sffamily}
\makeatother

\begin{document}
\pgfplotsset{compat=1.8}

\thispagestyle{empty}
\newpage
{
\centering


\textbf{
МОСКОВСКИЙ ГОСУДАРСТВЕННЫЙ ТЕХНИЧЕСКИЙ УНИВЕРСИТЕТ ИМЕНИ Н. Э. БАУМАНА \\
Факультет информатики и систем управления \\
Кафедра теоретической информатики и компьютерных технологий}
\bigskip
\bigskip
\bigskip
\bigskip
\bigskip
\bigskip
\bigskip

\vfill


Лабораторная работа №1 \\
по курсу <<Теория игр и исследование операций>>

\bigskip

{\large <<Линейное программирование. Симплекс-метод>>}
\bigskip

\vfill



\hfill\parbox{4cm} {
Выполнил:\\
студент ИУ9-111 \hfill \\
Выборнов А.И.\hfill \medskip\\
Руководитель:\\
Бассараб М.А.\hfill
}


\vspace{\fill}

Москва \number\year
\clearpage
}



\clearpage
% Отчет должен содержать: титульный лист; цель работы; постановку задачи; запись за-
% дачи ЛП в канонической форме; исходную симплекс-таблицу; промежуточные сим-
% плекс-таблицы; опорное решение; оптимальное решение; проверку решения на допу-
% стимость.

\section{Цель работы}
    Сформулировать задачу линейного программирования и решить её с помощью симплекс-метода.

\section{Постановка задачи}
    Найти вектор $x = [x_1, x_2, x_3]^T$ как решение следующей задачи:

    $$F = cx \rightarrow max,$$    $$Ax \leq b,$$    $$x_1, x_2, x_3 \geq 0.$$

    $$c = [2, 5, 3], 
    A = \begin{pmatrix}
        2 & 1 & 3\\
        1 & 2 & 0\\
        0 & 0.5 & 1\\
    \end{pmatrix}, 
    b^T = [6, 6, 2]$$

\section{Решение}

    Подставим числовые значения:
    \begin{gather}
        F = 2x_1+5x_2+3x_3 \rightarrow max,  \notag  \\
        \begin{cases}
            2x_1 + x_2 + 3x_3 \leq 6, \\
            x_1 + 2x_2 \leq 6, \\
            0.5x_2 + x_3 \leq 2.
        \end{cases} \notag \\
        x_1, x_2, x_3 \geq 0. \notag
    \end{gather} 

    Избавимся от неравенства - получим задачу в канонической форме:
    \begin{gather}
        F = 2x_1+5x_2+3x_3 \rightarrow max,  \notag  \\
        \begin{cases}
            2x_1 + x_2 + 3x_3 + x_4 = 6, \\
            x_1 + 2x_2 + x_5 = 6, \\
            0.5x_2 + x_3 + x_6 = 2.
        \end{cases} 
        \notag \\ x_1, x_2, x_3, x_4, x_5 \geq 0. \
    \end{gather} 

    Пусть $x_4,x_5,x_6$ --- базисные переменные, $x_1,x_2,x_3$ --- свободные переменные. Тогда имеем:
    \begin{gather}
        F = 2x_1+5x_2+3x_3 \rightarrow max,  \notag  \\
        \begin{cases}
            x_4 = 6 - (2x_1 + x_2 + 3x_3), \\
            x_5 = 6 - (x_1 + 2x_2),\\
            x_6 = 2 - (0.5x_2 + x_3).
        \end{cases} 
        \notag \\ x_1, x_2, x_3, x_4, x_5 \geq 0. \
    \end{gather} 

    Исходная симплекс-таблица записывается в виде:
    \begin{center}
        \begin{tabular}{|c|c|c|c|c|}
            \hline
                 & $s_{i0}$ & $x_1$ & $x_2$ & $x_3$ \\ \hline
            $x_4$ & 6       & 2     & 1     & 3 \\ \hline
            $x_5$ & 6       & 1     &{\bf 2}& 0 \\ \hline
            $x_6$ & 2       & 0     & 0.5   & 1 \\ \hline
            $F$   & 0       & -2    & -5    & -3 \\ \hline
        \end{tabular}
    \end{center}

    Так как в столбце свободных членов нет отрицательных элементов, то найдено опорное решение: $x=[0, 0, 0, 6, 6, 2], F(x)=0$. В строке F имеются отрицательные элементы, это означает что полученое решение не оптимально.

    $x_2$ --- разрешающий столбец, так как значение в строке таблицы, соответствующей целевой функции по модулю максимально.

    Найдем минимальное положительное отношение элемента свободных членов $s_{i0}$ к cоответствующем элементу в разрешающем столбце. Минимальное положительное отношение в строке $x_5$, выберем её в качестве разрешающей.

    Пересчитываем симплекс таблицу:
    \begin{center}
        \begin{tabular}{|c|c|c|c|c|}
            \hline
                 & $s_{i0}$ & $x_1$ & $x_5$ & $x_3$ \\ \hline
            $x_4$ & 3       & 1.5   & -0.5  & 3  \\ \hline
            $x_2$ & 3       & 0.5   & 0.5   & 0  \\ \hline
            $x_6$ & 0.5     & -0.25 & -0.25 &{\bf 1}\\ \hline
            $F$   & 15      & 0.5   & 2.5   & -3 \\ \hline
        \end{tabular}
    \end{center}

    В строке F имеются отрицательные элементы, это означает что полученое решение не оптимально.
    В качестве разрешающего столбца выбираем $x_3$ и в качестве разрешающей строки выбираем $x_6$ (причины выбора аналогичны описанным выше).

    Пересчитываем симплекс таблицу:
    \begin{center}
        \begin{tabular}{|c|c|c|c|c|}
            \hline
                 & $s_{i0}$ & $x_1$ & $x_5$ & $x_6$ \\ \hline
            $x_4$ & 1.5   &\bf{2.25}&0.25   & -3 \\ \hline
            $x_2$ & 3       & 0.5   & 0.5   & 0  \\ \hline
            $x_3$ & 0.5     & -0.25 & -0.25 & 1 \\ \hline
            $F$   & 16.5    & -0.25 & 1.75  & 3  \\ \hline
        \end{tabular}
    \end{center}

    В строке F имеются отрицательные элементы, это означает что полученое решение не оптимально.
    В качестве разрешающего столбца выбираем $x_1$ и в качестве разрешающей строки выбираем $x_4$ (причины выбора аналогичны описанным выше).

    Пересчитываем симплекс таблицу:
    \begin{center}
        \begin{tabular}{|c|c|c|c|c|}
            \hline
                 & $s_{i0}$ & $x_4$ & $x_5$ & $x_6$ \\ \hline
            $x_1$ & 0.6(6)    &0.4(4) & 0.1(1)  & -1.3(3) \\ \hline
            $x_2$ & 2.6(6)    & -0.2(2) & 0.4(4)  & 0.6(6) \\ \hline
            $x_3$ & 0.6(6)    & 0.1(1)  & -0.2(2) & 0.6(6) \\ \hline
            $F$   & 16.6(6)   & 0.1(1)  & 1.7(7)  & 2.6(6) \\ \hline
        \end{tabular}
    \end{center}

    Среди значений индексной строки нет отрицательных. Поэтому эта таблица определяет оптимальное решение:
    $$x = [\frac{2}{3}, 2\frac{2}{3}, \frac{2}{3}],$$ 
    $$max(F(x)) = 16\frac{2}{3}.$$

    Проверим полученное решение на допустимость:
    \begin{gather}
        \begin{cases}
            x_1 = \frac{2}{3},\\
            x_2 = 2\frac{2}{3}, \\
            x_3 = \frac{2}{3}, \\
            x_4 = 6 - (2x_1 + x_2 + 3x_3) = 6 - (4 + 2)=0, \\
            x_5 = 6 - (x_1 + 2x_2) = 6 - 6 = 0,\\
            x_6 = 2 - (0.5x_2 + x_3) = 2 - 2= 0.
        \end{cases} 
    \end{gather}
    Решение допустимое, так как все переменные неотрицательны.
\end{document}