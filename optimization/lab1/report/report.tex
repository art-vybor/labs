
% utf-8 ru, unix eolns
\documentclass[12pt,a4paper,oneside]{extarticle}
    \righthyphenmin=2 %минимально переносится 2 символа %%%
    \sloppy

% Рукопись оформлена в соответствии с правилами оформления 
% электронной версии авторского оригинала, 
% принятыми в Издательстве МГТУ им. Н.Э. Баумана.

\usepackage{geometry} % А4, примерно 28-31 строк(а) на странице 
    \geometry{paper=a4paper}
    \geometry{includehead=false} % Нет верх. колонтитула
    \geometry{includefoot=true}  % Есть номер страницы
    \geometry{bindingoffset=0mm} % Переплет    : 0  мм
    \geometry{top=20mm}          % Поле верхнее: 20 мм
    \geometry{bottom=25mm}       % Поле нижнее : 25 мм 
    \geometry{left=25mm}         % Поле левое  : 25 мм
    \geometry{right=25mm}        % Поле правое : 25 мм
    \geometry{headsep=10mm}  % От края до верх. колонтитула: 10 мм
    \geometry{footskip=20mm} % От края до нижн. колонтитула: 20 мм 

\usepackage{cmap}
\usepackage[T2A]{fontenc} 
\usepackage[utf8x]{inputenc}
\usepackage[english,russian]{babel}
\usepackage{misccorr}

\usepackage{amsmath}
\usepackage{amsfonts}
\usepackage{amssymb}

%\usepackage{cm-super} %человеческий рендер русских шрифтов

\setlength{\parindent}{1.25cm}  % Абзацный отступ: 1,25 см
\usepackage{indentfirst}        % 1-й абзац имеет отступ

\usepackage{setspace}   

\onehalfspacing % Полуторный интервал между строками

\makeatletter
\renewcommand{\@oddfoot }{\hfil\thepage\hfil} % Номер стр.
\renewcommand{\@evenfoot}{\hfil\thepage\hfil} % Номер стр.
\renewcommand{\@oddhead }{} % Нет верх. колонтитула
\renewcommand{\@evenhead}{} % Нет верх. колонтитула
\makeatother

\usepackage{fancyvrb}


\usepackage[pdftex]{graphicx}  % поддержка картинок для пдф
\graphicspath{ {./pictures/} }
\usepackage{rotating}
%\DeclareGraphicsExtensions{.jpg,.png}




\renewcommand{\labelenumi}{\theenumi.} %меняет вид нумерованного списка

\usepackage{perpage} %нумерация сносок 
\MakePerPage{footnote}

\usepackage[all]{xy} %поддержка графов

\usepackage{listings} %листинги
\renewcommand{\lstlistingname}{Листинг}
\lstset{
  basicstyle=\tiny,
  breaklines=true
  }


\usepackage{url}


\usepackage{tikz} %для рисования графиков
\usepackage{pgfplots}

\usepackage{gensymb}

\usepackage{ccaption}%изменяет подпись к рисунку
\makeatletter 
\renewcommand{\fnum@figure}[1]{Рисунок~\thefigure~---~\sffamily}
\makeatother

\begin{document}
\pgfplotsset{compat=1.8}

\thispagestyle{empty}
\newpage
{
\centering


\textbf{
МОСКОВСКИЙ ГОСУДАРСТВЕННЫЙ ТЕХНИЧЕСКИЙ УНИВЕРСИТЕТ ИМЕНИ Н. Э. БАУМАНА \\
Факультет информатики и систем управления \\
Кафедра теоретической информатики и компьютерных технологий}
\bigskip
\bigskip
\bigskip
\bigskip
\bigskip
\bigskip
\bigskip

\vfill


Лабораторная работа №1 \\
по курсу <<Методы оптимизации>>

\bigskip

{\large <<Необходимые и достаточные условия существования безусловного и условного экстремума>>}
\bigskip

\vfill



\hfill\parbox{4cm} {
Выполнил:\\
студент ИУ9-111 \hfill \\
Выборнов А. И.\hfill \medskip\\
Руководитель:\\
Каганов Ю. Т.\hfill
}


\vspace{\fill}

Москва \number\year
\clearpage
}



\clearpage

\section{Постановка задачи}
    \subsection {Задача 1}
        \begin{enumerate}
            \item Найти экстремум функции $f(x) = x_1^3 + x_2^2 + x_3^2 - 3*x_1 + x_2*x_3 + 6*x_2 + 2$. 
            \item Записать необходимые условия экстремума первого порядка.
            \item Проверить выполнение достаточных и необходимых условий второго порядка в каждой стационарной точке двумя способами.
            \item Найти все стационарные точки и значения функций соответствующие этим точкам.
        \end{enumerate}

    \subsection {Задача 2}
        \begin{enumerate}
            \item Найти экстремум функции
            \begin{equation*}
                \begin{cases}
                   f(x) = (x^3 - \alpha)^2 + x_2^2 \rightarrow extr, \\
                   g_1(x) = x_1^2+x_2^2 - 1 \le 0, \\
                   g_2(x) = -x_1 \le 0.
                \end{cases}
            \end{equation*}
            \item Записать необходимые условия экстремума первого порядка.
            \item Проверить выполнение достаточных и необходимых условий второго порядка в каждой стационарной точке двумя способами.
            \item Найти все стационарные точки и значения функций соответствующие этим точкам.
        \end{enumerate}
\section{Решение}
    \subsection {Задача 1}
        Необходимые условия экстремума первого порядка:
        $\nabla f(x) = 0$

        В результате решения системы для функции $f(x) = x_1^3 + x_2^2 + x_3^2 - 3*x_1 + x_2*x_3 + 6*x_2 + 2$, находим две стационарные точки:
        \begin{itemize}
            \item $x^l_1 = (0.99999994, -4.00000028,  2.00000014)$
            \item $x^l_2 = (1.35948563e+08,   2.41531279e+10,  -2.43288845e+08)$
        \end{itemize}

        

\end{document}
