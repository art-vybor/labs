
% utf-8 ru, unix eolns
\documentclass[12pt,a4paper,oneside]{extarticle}
    \righthyphenmin=2 %минимально переносится 2 символа %%%
    \sloppy

% Рукопись оформлена в соответствии с правилами оформления 
% электронной версии авторского оригинала, 
% принятыми в Издательстве МГТУ им. Н.Э. Баумана.

\usepackage{geometry} % А4, примерно 28-31 строк(а) на странице 
    \geometry{paper=a4paper}
    \geometry{includehead=false} % Нет верх. колонтитула
    \geometry{includefoot=true}  % Есть номер страницы
    \geometry{bindingoffset=0mm} % Переплет    : 0  мм
    \geometry{top=20mm}          % Поле верхнее: 20 мм
    \geometry{bottom=25mm}       % Поле нижнее : 25 мм 
    \geometry{left=25mm}         % Поле левое  : 25 мм
    \geometry{right=25mm}        % Поле правое : 25 мм
    \geometry{headsep=10mm}  % От края до верх. колонтитула: 10 мм
    \geometry{footskip=20mm} % От края до нижн. колонтитула: 20 мм 

\usepackage{cmap}
\usepackage[T2A]{fontenc} 
\usepackage[utf8x]{inputenc}
\usepackage[english,russian]{babel}
\usepackage{misccorr}

\usepackage{amsmath}
\usepackage{amsfonts}
\usepackage{amssymb}

%\usepackage{cm-super} %человеческий рендер русских шрифтов

\setlength{\parindent}{1.25cm}  % Абзацный отступ: 1,25 см
\usepackage{indentfirst}        % 1-й абзац имеет отступ

\usepackage{setspace}   

\onehalfspacing % Полуторный интервал между строками

\makeatletter
\renewcommand{\@oddfoot }{\hfil\thepage\hfil} % Номер стр.
\renewcommand{\@evenfoot}{\hfil\thepage\hfil} % Номер стр.
\renewcommand{\@oddhead }{} % Нет верх. колонтитула
\renewcommand{\@evenhead}{} % Нет верх. колонтитула
\makeatother

\usepackage{fancyvrb}


\usepackage[pdftex]{graphicx}  % поддержка картинок для пдф
\graphicspath{ {./pictures/} }
\usepackage{rotating}
%\DeclareGraphicsExtensions{.jpg,.png}




\renewcommand{\labelenumi}{\theenumi.} %меняет вид нумерованного списка

\usepackage{perpage} %нумерация сносок 
\MakePerPage{footnote}

\usepackage[all]{xy} %поддержка графов

\usepackage{listings} %листинги
\renewcommand{\lstlistingname}{Листинг}
\lstset{
  basicstyle=\tiny,
  breaklines=true
  }


\usepackage{url}


\usepackage{tikz} %для рисования графиков
\usepackage{pgfplots}

\usepackage{gensymb}

\usepackage{ccaption}%изменяет подпись к рисунку
\makeatletter 
\renewcommand{\fnum@figure}[1]{Рисунок~\thefigure~---~\sffamily}
\makeatother

\begin{document}
\pgfplotsset{compat=1.8}

\thispagestyle{empty}
\newpage
{
\centering


\textbf{
МОСКОВСКИЙ ГОСУДАРСТВЕННЫЙ ТЕХНИЧЕСКИЙ УНИВЕРСИТЕТ ИМЕНИ Н. Э. БАУМАНА \\
Факультет информатики и систем управления \\
Кафедра теоретической информатики и компьютерных технологий}
\bigskip
\bigskip
\bigskip
\bigskip
\bigskip
\bigskip
\bigskip

\vfill


Лабораторная работа №1 \\
по курсу <<Методы оптимизации>>

\bigskip

{\large <<Необходимые и достаточные условия существования безусловного и условного экстремума>>}
\bigskip

\vfill



\hfill\parbox{4cm} {
Выполнил:\\
студент ИУ9-111 \hfill \\
Выборнов А. И.\hfill \medskip\\
Руководитель:\\
Каганов Ю. Т.\hfill
}


\vspace{\fill}

Москва \number\year
\clearpage
}



\clearpage

\section{Задача 1}
    \subsection{Постановка задачи}
        Найти экстремум функции $f(x) = x_1^3 + x_2^2 + x_3^2 - 3*x_1 + x_2*x_3 + 6*x_2 + 2$. 

    \subsection{Решение}
        Необходимые условия экстремума первого порядка:
        $$\nabla f(x) = 0,$$ где $f(x) = x_1^3 + x_2^2 + x_3^2 - 3*x_1 + x_2*x_3 + 6*x_2 + 2$. Получаем систему уравнений:
        \begin{equation}
            \begin{cases}
               3x_1^2-3 = 0, \\
               2x_2 + x_3 + 6 = 0, \\
               2x_3 + x2 = 0.
            \end{cases}
        \end{equation}

        В результате решения вышеприведённой системы, находим две стационарные точки: $x^l_1 = (1, -4, 2)$ и $x^l_2 = (-1, -4, 2)$.

        Запищем матрицу Гессе для исходной функции:
        $$
        H=\begin{bmatrix}
            6x_1 & 0 &0 \\
            0 & 2 & 1 \\
            0 & 1 & 2
        \end{bmatrix}.
        $$

        Проверяем выполнение достаточного условия для точки $x^l_1$:
        \begin{equation}
            \begin{cases}
               \Delta_1 = 6 > 0, \\
               \Delta_2 = 12 > 0, \\
               \Delta_3 = 18 > 0.
            \end{cases}
        \end{equation}
        Получили, что матрица Гессе $H(x^l_1)$ положительно определена и точка $x^l_1$ является точкой локального минимума.
        
        Проверяем выполнение достаточного условия для точки $x^l_2$:
        \begin{equation}
            \begin{cases}
               \Delta_1 = -6 < 0, \\
               \Delta_2 = -12 < 0, \\
               \Delta_3 = -18 < 0.
            \end{cases}
        \end{equation}
        Получили, что для точки $x^l_2$ достаточное условие не выполняется.

        Проверяем выполнение необходимого условия второго порядка для точки $x^l_2$. Критерий проверки необходимых условий экстремума второго порядка не выполняется. Следовательно $x^l_2$ - не точка экстремума.

        Нашли точку экстремума $x^l_1 = (1, -4, 2)$, которая является точкой локального минимума и $f(x^l_1) = -12$.

\section{Задача 2}
    \subsection{Постановка задачи}
        Найти экстремум функции $f(x)$ (подобрать $\alpha$):
        \begin{equation*}
            \begin{cases}
               f(x) = (x_1^3 - \alpha)^2 + x_2^2 \rightarrow extr, \\
               g_1(x) = x_1^2+x_2^2 - 1 \le 0, \\
               g_2(x) = -x_1 \le 0.
            \end{cases}
        \end{equation*}

    \subsection{Решение}
        Составим обобщённую функцию Лагранжа, которая имеет вид:
        $$L(x,\lambda_0, \lambda) = \lambda_0f(x)+\sum_{j=1}^2\lambda_jg_j(x).$$

        Получим необходимые условия экстремума первого порядка. Условие стационарности обощённой функции:
        $$\frac{\delta L(x,\lambda_0, \lambda)}{\delta x_i}=0, i=1,2.$$
        $$\frac{\delta L}{\delta x_1} + \frac{\delta L}{\delta x_2} = 6\lambda_0(x_1^3 - \alpha)x_1^2 + 2\lambda_1x_1 - \lambda_2 +2\lambda_0x_2 + 2\lambda_1x_2 = 0.$$
        Условие допустимости решения:
        \begin{equation*}
            \begin{cases}
               g_1(x) = x_1^2+x_2^2 - 1 \le 0, \\
               g_2(x) = -x_1 \le 0.
            \end{cases}
        \end{equation*}
        Условия неотрицательности и неположительности не рассматриваются в виду условия задачи. Условие дополняющей нежёсткости:
        \begin{equation*}
            \begin{cases}
                \lambda_1(x_1^2+x_2^2 - 1) = 0, \\
                \lambda_2(-x_1) = 0.
            \end{cases}
        \end{equation*}
        Получили систему уравнений:
        \begin{equation*}
            \begin{cases}
                6\lambda_0(x_1^3 - \alpha)x_1^2 + 2\lambda_1x_1 - \lambda_2 +2\lambda_0x_2 + 2\lambda_1x_2 = 0, \\
                x_1^2+x_2^2 - 1 \le 0, \\
                -x_1 \le 0, \\
                \lambda_1(x_1^2+x_2^2 - 1) = 0, \\
                \lambda_2(-x_1) = 0.
            \end{cases}
        \end{equation*}

        Рассмотрим два случая:
        \begin{enumerate}
            \item $\lambda_0 = 0$,
            \item $\lambda_0 \neq 0$, при этом $\frac{\lambda_j}{\lambda_0} = \lambda_j'$.
        \end{enumerate}

        \subsubsection{Случай 1 ($\lambda_0 = 0$)}
            \begin{equation*}
                \begin{cases}
                    2\lambda_1x_1 - \lambda_2 + 2\lambda_1x_2 = 0, \\
                    x_1^2+x_2^2 - 1 \le 0, \\
                    -x_1 \le 0, \\
                    \lambda_1(x_1^2+x_2^2 - 1) = 0, \\
                    \lambda_2(-x_1) = 0.
                \end{cases}
            \end{equation*}

            Пусть $\lambda_1 \neq 0, \lambda_2 \neq 0$, тогда получаем два решения $x^l = (0, \pm 1)$. Пусть $\lambda_1 = 0$, тогда $\lambda_2 = 0$ и получаем систему решений:
            \begin{equation*}
                \begin{cases}
                    (x^l_1)^2+(x^l_2)^2 - 1 \le 0, \\
                    x^l_1 \in [0, 1], \\
                    x^l_2 \in [-1,1].
                \end{cases}
            \end{equation*}
    \subsection{Проверка достаточных условий экстремума}
        Проверим достаточные условия экстремума для точек, полученных на предыдущем этапе ($\lambda_0=0, \lambda_1 \neq 0, \lambda_2 \neq 0$). Для этого необходимо $d^2L(x^l, \lambda^l) > (<) 0$.

        Получаем $d^2L(x^l, \lambda^l) = 4\lambda_1 \neq 0$. То есть, точки $x^l = (0, \pm 1)$ является точками локальных экстремумов.

    \subsection{Вычисление значений функции в точках экстремума}
        $$f([0,1]) = \alpha^2 + 1$$
        $$f([0,-1]) = \alpha^2 + 1$$

        % \subsubsection{Случай 2 ($\lambda_0 \neq 0$)}
        %     Для простоты положим $\lambda_0 = 1$ (можно использовать подобное упрощение, так как если $\lambda_0$ отличен от 1, то мы можем поделить всё уравнение на $\lambda_0$).
        %     \begin{equation*}
        %         \begin{cases}
        %             6(x_1^3 - \alpha)x_1^2 + 2\lambda_1x_1 - \lambda_2 +2x_2 + 2\lambda_1x_2 = 0, \\
        %             x_1^2+x_2^2 - 1 \le 0, \\
        %             -x_1 \le 0, \\
        %             \lambda_1(x_1^2+x_2^2 - 1) = 0, \\
        %             \lambda_2(-x_1) = 0.
        %         \end{cases}
        %     \end{equation*}

        %     Пусть $\lambda_1 \neq 0, \lambda_2 \neq 0$, тогда получаем два решения $x^l = (0, \pm 1)$. Пусть $\lambda_1 = 0, \lambda_2 \neq 0$, тогда получаем систему решений: 
        %     \begin{equation*}
        %         \begin{cases}
        %             x^l_1 = 0, \\
        %             x^l_2 \in [-1,1].
        %         \end{cases}
        %     \end{equation*}

        %     Пусть $\lambda_1 \neq 0, \lambda_2 = 0$, тогда получаем систему решений:
        %     \begin{equation*}
        %         \begin{cases}
        %             6(x_1^3 - \alpha)x_1^2 + 2\lambda_1x_1 +2x_2 + 2\lambda_1x_2 = 0,\\
        %             (x^l_1)^2+(x^l_2)^2 = 1, \\
        %             x^l_1 \in [0,1], \\
        %             x^l_2 \in [-1,1].
        %         \end{cases}
        %     \end{equation*}

\end{document}
