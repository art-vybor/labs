
% utf-8 ru, unix eolns
\documentclass[12pt,a4paper,oneside]{extarticle}
    \righthyphenmin=2 %минимально переносится 2 символа %%%
    \sloppy

% Рукопись оформлена в соответствии с правилами оформления 
% электронной версии авторского оригинала, 
% принятыми в Издательстве МГТУ им. Н.Э. Баумана.

\usepackage{geometry} % А4, примерно 28-31 строк(а) на странице 
    \geometry{paper=a4paper}
    \geometry{includehead=false} % Нет верх. колонтитула
    \geometry{includefoot=true}  % Есть номер страницы
    \geometry{bindingoffset=0mm} % Переплет    : 0  мм
    \geometry{top=20mm}          % Поле верхнее: 20 мм
    \geometry{bottom=25mm}       % Поле нижнее : 25 мм 
    \geometry{left=25mm}         % Поле левое  : 25 мм
    \geometry{right=25mm}        % Поле правое : 25 мм
    \geometry{headsep=10mm}  % От края до верх. колонтитула: 10 мм
    \geometry{footskip=20mm} % От края до нижн. колонтитула: 20 мм 

\usepackage{cmap}
\usepackage[T2A]{fontenc} 
\usepackage[utf8x]{inputenc}
\usepackage[english,russian]{babel}
\usepackage{misccorr}

\usepackage{amsmath}
\usepackage{amsfonts}
\usepackage{amssymb}

%\usepackage{cm-super} %человеческий рендер русских шрифтов

\setlength{\parindent}{1.25cm}  % Абзацный отступ: 1,25 см
\usepackage{indentfirst}        % 1-й абзац имеет отступ

\usepackage{setspace}   

\onehalfspacing % Полуторный интервал между строками

\makeatletter
\renewcommand{\@oddfoot }{\hfil\thepage\hfil} % Номер стр.
\renewcommand{\@evenfoot}{\hfil\thepage\hfil} % Номер стр.
\renewcommand{\@oddhead }{} % Нет верх. колонтитула
\renewcommand{\@evenhead}{} % Нет верх. колонтитула
\makeatother

\usepackage{fancyvrb}


\usepackage[pdftex]{graphicx}  % поддержка картинок для пдф
\graphicspath{ {./pictures/} }
\usepackage{rotating}
%\DeclareGraphicsExtensions{.jpg,.png}




\renewcommand{\labelenumi}{\theenumi.} %меняет вид нумерованного списка

\usepackage{perpage} %нумерация сносок 
\MakePerPage{footnote}

\usepackage[all]{xy} %поддержка графов

\usepackage{listings} %листинги
\renewcommand{\lstlistingname}{Листинг}
\lstset{
  basicstyle=\tiny,
  breaklines=true
  }


\usepackage{url}


\usepackage{tikz} %для рисования графиков
\usepackage{pgfplots}

\usepackage{gensymb}

\usepackage{ccaption}%изменяет подпись к рисунку
\makeatletter 
\renewcommand{\fnum@figure}[1]{Рисунок~\thefigure~---~\sffamily}
\makeatother

\begin{document}
\pgfplotsset{compat=1.8}

\thispagestyle{empty}
\newpage
{
\centering


\textbf{
МОСКОВСКИЙ ГОСУДАРСТВЕННЫЙ ТЕХНИЧЕСКИЙ УНИВЕРСИТЕТ ИМЕНИ Н. Э. БАУМАНА \\
Факультет информатики и систем управления \\
Кафедра теоретической информатики и компьютерных технологий}
\bigskip
\bigskip
\bigskip
\bigskip
\bigskip
\bigskip
\bigskip

\vfill


Лабораторная работа №5 \\
по курсу <<Методы оптимизации>>

\bigskip

{\large <<Решение задач многоэкстримальной оптимизации на основе популяционных алгоритмов>>}
\bigskip

\vfill



\hfill\parbox{4cm} {
Выполнил:\\
студент ИУ9-111 \hfill \\
Выборнов А. И.\hfill \medskip\\
Руководитель:\\
Каганов Ю. Т.\hfill
}


\vspace{\fill}

Москва \number\year
\clearpage
}



\clearpage

\section{Метод дифференциальной эволюции (вариант генетического алгоритма)}
    \subsection{Постановка задачи}
        Дана функция Швефеля:
        $$f(x_1, x_2) = ax_1 \sin(b\sqrt|x_1|) + x_2 \sin(c\sqrt|x_2|),$$ 
        определённая на множестве допустимых решений: $[x_{min}, x_{max}] = [-500,500]$.
        Исследовать характер решений в зависимости от параметров $a,b,c$ при значениях: 
        $a = (0.5, 1.0, 2.5), b = (0.5, 0.8, 1.0), c = (0.5, 1.0, 2.0)$, а именно найти глобальный экстремум.
        

    \subsection{Решение на языке программирования python}
        \lstinputlisting{../doit.py}

    \subsection{Результат работы}
        Результат работы программы при разных параметрах приведён в таблице:
        \begin{figure} [ht]
            \centering
            \begin{tabular}{ |c|c|c|c|c| }
                \hline
                a & b & c & [x1, x2] & f([x1, x2]) \\ \hline
                0.5 & 0.5 & 0.5 & [499.31742395249955, 499.3174262797162] & -737.257041083 \\ \hline
                1.0 & 0.8 & 1.0 & [472.6952114926131, -420.968746351129] & -888.583750801 \\ \hline
                2.5 & 1.0 & 2.0 & [-420.9687469299634, 450.6825579994172] & -1497.64060671 \\ \hline
            \end{tabular}
        \end{figure}

\end{document}