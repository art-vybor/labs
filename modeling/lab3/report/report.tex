
% utf-8 ru, unix eolns
\documentclass[12pt,a4paper,oneside]{extarticle}
    \righthyphenmin=2 %минимально переносится 2 символа %%%
    \sloppy

% Рукопись оформлена в соответствии с правилами оформления 
% электронной версии авторского оригинала, 
% принятыми в Издательстве МГТУ им. Н.Э. Баумана.

\usepackage{geometry} % А4, примерно 28-31 строк(а) на странице 
    \geometry{paper=a4paper}
    \geometry{includehead=false} % Нет верх. колонтитула
    \geometry{includefoot=true}  % Есть номер страницы
    \geometry{bindingoffset=0mm} % Переплет    : 0  мм
    \geometry{top=20mm}          % Поле верхнее: 20 мм
    \geometry{bottom=25mm}       % Поле нижнее : 25 мм 
    \geometry{left=25mm}         % Поле левое  : 25 мм
    \geometry{right=25mm}        % Поле правое : 25 мм
    \geometry{headsep=10mm}  % От края до верх. колонтитула: 10 мм
    \geometry{footskip=20mm} % От края до нижн. колонтитула: 20 мм 

\usepackage{cmap}
\usepackage[T2A]{fontenc} 
\usepackage[utf8x]{inputenc}
\usepackage[english,russian]{babel}
\usepackage{misccorr}

\usepackage{amsmath}
\usepackage{amsfonts}
\usepackage{amssymb}

%\usepackage{cm-super} %человеческий рендер русских шрифтов

\setlength{\parindent}{1.25cm}  % Абзацный отступ: 1,25 см
\usepackage{indentfirst}        % 1-й абзац имеет отступ

\usepackage{setspace}   

\onehalfspacing % Полуторный интервал между строками

\makeatletter
\renewcommand{\@oddfoot }{\hfil\thepage\hfil} % Номер стр.
\renewcommand{\@evenfoot}{\hfil\thepage\hfil} % Номер стр.
\renewcommand{\@oddhead }{} % Нет верх. колонтитула
\renewcommand{\@evenhead}{} % Нет верх. колонтитула
\makeatother

\usepackage{fancyvrb}


\usepackage[pdftex]{graphicx}  % поддержка картинок для пдф
\graphicspath{ {./pictures/} }
\usepackage{rotating}
%\DeclareGraphicsExtensions{.jpg,.png}




\renewcommand{\labelenumi}{\theenumi.} %меняет вид нумерованного списка

\usepackage{perpage} %нумерация сносок 
\MakePerPage{footnote}

\usepackage[all]{xy} %поддержка графов

\usepackage{listings} %листинги
\renewcommand{\lstlistingname}{Листинг}
\lstset{
  basicstyle=\small,
  breaklines=true
  }


\usepackage{url}


\usepackage{tikz} %для рисования графиков
\usepackage{pgfplots}

\usepackage{gensymb}

\usepackage{ccaption}%изменяет подпись к рисунку
\makeatletter 
\renewcommand{\fnum@figure}[1]{Рисунок~\thefigure~---~\sffamily}
\makeatother

\begin{document}
\pgfplotsset{compat=1.8}

\thispagestyle{empty}
\newpage
{
\centering


\textbf{
МОСКОВСКИЙ ГОСУДАРСТВЕННЫЙ ТЕХНИЧЕСКИЙ УНИВЕРСИТЕТ ИМЕНИ Н. Э. БАУМАНА \\
Факультет информатики и систем управления \\
Кафедра теоретической информатики и компьютерных технологий}
\bigskip
\bigskip
\bigskip
\bigskip
\bigskip
\bigskip
\bigskip

\vfill


Лабораторная работа №3 \\
по курсу <<Математическое моделирование>>

\bigskip

{\large <<Определение корреляции между наборами данных>>}
\bigskip

\vfill



\hfill\parbox{4cm} {
Выполнил:\\
студент ИУ9-111 \hfill \\
Выборнов А. И.\hfill \medskip\\
Руководитель:\\
Домрачева А. Б.\hfill
}


\vspace{\fill}

Москва \number\year
\clearpage
}



\clearpage


\section{Постановка задачи}
    Рассматриваются 6 станций чешского метрополитена. Для каждой станции, вручную была посчитана следующая информация с точностью до месяца:
    \begin{itemize}
        \item Среднее число пассажиров, вошедших с данной станции в метрополитена в день.
        \item Среднее число пассажиров, вышедших с данной станции в день.
    \end{itemize}

    Данные для 6 станций (A0, A1, B0, B1, C0, C1) приведены в таблице~\ref{tabular:data}. Строки соответствуют месяцам, столбцы станциям метро, причём префикс "th" \space соответствует вошедшим пассажирам, а префикс "r" \space вышедшим.
    \begin{table}[ht!]
        \caption{Данные о числе пассажиров проходящих через станции Пражского метро}
        \centering
        \label{tabular:data}
        \small
        \begin{tabular}{|c|c|c|c|c|c|c|c|c|c|c|c|c|}
        \hline
        m/s & thA0 & rA0 & thA1 & rA1 & thB0 & rB0 & thB1 & rB1 & thC0 & rC0 & thC1 & rC1 \\ \hline
        1 & 16551 & 14899 & 30746 & 27320 & 32822 & 29553 & 21002 & 18793 & 17084 & 15365 & 4544 & 3118 \\ \hline
        2 & 16810 & 14292 & 22558 & 20155 & 25314 & 22567 & 40022 & 35436 & 29096 & 25876 & 17519 & 16162 \\ \hline
        3 & 14434 & 13046 & 28001 & 24916 & 36918 & 32720 & 35118 & 31145 & 38639 & 34226 & 38841 & 34819 \\ \hline
        4 & 20891 & 18696 & 32958 & 29255 & 46677 & 41259 & 20283 & 18164 & 23690 & 21145 & 37324 & 33492 \\ \hline
        5 & 13773 & 12468 & 28277 & 25159 & 16909 & 15212 & 41746 & 36944 & 29087 & 25868 & 16717 & 15461 \\ \hline
        6 & 14739 & 13313 & 36763 & 32398 & 21889 & 19569 & 40458 & 35817 & 21993 & 20494 & 40099 & 35920 \\ \hline
        7 & 24713 & 22040 & 34650 & 30735 & 34998 & 31040 & 19478 & 17460 & 30082 & 26738 & 42244 & 37797 \\ \hline
        8 & 10127 & 9278 & 33590 & 29808 & 23285 & 20791 & 22974 & 21353 & 18776 & 17263 & 22099 & 20170 \\ \hline
        9 & 14689 & 13269 & 12239 & 11126 & 21561 & 19282 & 25348 & 23430 & 34808 & 31290 & 40895 & 36617 \\ \hline
        10 & 13047 & 11833 & 35848 & 31784 & 37778 & 33472 & 25336 & 22586 & 26192 & 23751 & 17519 & 16162 \\ \hline
        11 & 16487 & 14843 & 38451 & 34061 & 29376 & 26120 & 23743 & 22025 & 18230 & 16784 & 38841 & 34819 \\ \hline
        12 & 14345 & 12968 & 18573 & 16668 & 32822 & 29553 & 29751 & 27282 & 37085 & 33283 & 37324 & 33492 \\ \hline
        \end{tabular}
    \end{table}

    Необходимо найти корреляцию между числом вошедших и вышедших со станции.

\section{Решение}
    Для каждой станции метро обрабатывались данные за последние 12 месяцев. Для этих данных были найдены следующие величины:
    \begin{itemize}
        \item Коэффициент корреляции Пирсона,
        \item Ранговый коэффициент корреляции Пирсона,
        \item Ранговый коэффициент Спирмана.
    \end{itemize}

    Вычисления производились с помощью скрипта на языке python. Фрагмент скрипта, вычисляющий корреляцию:
    \begin{lstlisting}
def get_rank_table(X):
    start, end = min(X), max(X)
    step = (end - start)/9

    rank_X = {x: int((x - start)/step + 1) for x in X}
    return [rank_X[x] for x in X]

def rank_spirmen(X,Y):
    rank_X = get_rank_table(X)
    rank_Y = get_rank_table(Y)

    sum_d = [(x-y)**2 for x,y in zip(rank_X, rank_Y)]
    return 1 - 6.0*sum(sum_d)/(len(X)**3 - len(X))

def linear_pirson(X,Y):
    return cov(X,Y)*1.0/(math.sqrt(D(X))*math.sqrt(D(Y)))

def rank_pirson(X,Y):
    rank_X = get_rank_table(X)
    rank_Y = get_rank_table(Y)
    return sum([(x-y)**2*1.0/y for x,y in zip(rank_X, rank_Y)])
    \end{lstlisting}
    
    Результаты вычисления коэффициентов корреляции занесены в таблицу~\ref{tabular:syntax}.
    \begin{table}[ht!]
        \caption{Значения различных коэффицентов корреляции для каждой из станций метро.}
        \centering
        \label{tabular:syntax}
        \begin{tabular}{|c|c|c|c|c|c|c|}
            \hline
                                                    & A0    & A1    & B0    & B1    & C0    & C 1 \\ \hline
            Коэффициент корреляции Пирсона          & 0.997 & 0.999 & 0.999 & 0.998 & 0.999 & 0.999 \\ \hline
            Ранговый коэффициент корреляции Пирсона & 0.25  & 0.0   & 0.0   & 0.333 & 0.2   & 0.25 \\ \hline
            Ранговый коэффициент Спирмана           & 0.996 & 1.0   & 1.0   & 0.996 & 0.996 & 0.996 \\ \hline
        \end{tabular}
    \end{table}

    Как видно из таблицы, все коэффициенты показали высокую степень корреляции между данными. То что коэффициент корреляции Пирсона и ранговый коэффициент Спирмана показали значения близкие к единице, показывает наличие линейной зависимости между величинами. Показать, что попадают в интервалы. Для анализа значения рангового коэффициента корреляции Пирсона необходимо провести дополнительное исследование, но в рамках данной работы оно было опущенно, так как мы получили очень малые результаты близкие к нулю, что означает наличие высокой корреляции между данными.
\end{document}
