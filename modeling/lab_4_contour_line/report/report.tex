% utf-8 ru, unix eolns
\documentclass[12pt,a4paper,oneside]{extarticle}
    \righthyphenmin=2 %минимально переносится 2 символа %%%
    \sloppy

% Рукопись оформлена в соответствии с правилами оформления 
% электронной версии авторского оригинала, 
% принятыми в Издательстве МГТУ им. Н.Э. Баумана.

\usepackage{geometry} % А4, примерно 28-31 строк(а) на странице 
    \geometry{paper=a4paper}
    \geometry{includehead=false} % Нет верх. колонтитула
    \geometry{includefoot=true}  % Есть номер страницы
    \geometry{bindingoffset=0mm} % Переплет    : 0  мм
    \geometry{top=20mm}          % Поле верхнее: 20 мм
    \geometry{bottom=25mm}       % Поле нижнее : 25 мм 
    \geometry{left=25mm}         % Поле левое  : 25 мм
    \geometry{right=25mm}        % Поле правое : 25 мм
    \geometry{headsep=10mm}  % От края до верх. колонтитула: 10 мм
    \geometry{footskip=20mm} % От края до нижн. колонтитула: 20 мм 

\usepackage{cmap}
\usepackage[T2A]{fontenc} 
\usepackage[utf8x]{inputenc}
\usepackage[english,russian]{babel}
\usepackage{misccorr}

\usepackage{amsmath}
\usepackage{amsfonts}
\usepackage{amssymb}

%\usepackage{cm-super} %человеческий рендер русских шрифтов

\setlength{\parindent}{1.25cm}  % Абзацный отступ: 1,25 см
\usepackage{indentfirst}        % 1-й абзац имеет отступ

\usepackage{setspace}   

\onehalfspacing % Полуторный интервал между строками

\makeatletter
\renewcommand{\@oddfoot }{\hfil\thepage\hfil} % Номер стр.
\renewcommand{\@evenfoot}{\hfil\thepage\hfil} % Номер стр.
\renewcommand{\@oddhead }{} % Нет верх. колонтитула
\renewcommand{\@evenhead}{} % Нет верх. колонтитула
\makeatother

\usepackage{fancyvrb}


\usepackage[pdftex]{graphicx}  % поддержка картинок для пдф
\graphicspath{ {./pictures/} }
\usepackage{rotating}
%\DeclareGraphicsExtensions{.jpg,.png}




\renewcommand{\labelenumi}{\theenumi.} %меняет вид нумерованного списка

\usepackage{perpage} %нумерация сносок 
\MakePerPage{footnote}

\usepackage[all]{xy} %поддержка графов

\usepackage{listings} %листинги


\usepackage{url}


\usepackage{tikz} %для рисования графиков
\usepackage{pgfplots}

\usepackage{gensymb}

\usepackage{ccaption}%изменяет подпись к рисунку
\makeatletter 
\renewcommand{\fnum@figure}[1]{Рисунок~\thefigure~---~\sffamily}
\makeatother

\begin{document}
\pgfplotsset{compat=1.8}

\thispagestyle{empty}
\newpage
{
\centering


\textbf{
МОСКОВСКИЙ ГОСУДАРСТВЕННЫЙ ТЕХНИЧЕСКИЙ УНИВЕРСИТЕТ ИМЕНИ Н. Э. БАУМАНА \\
Факультет информатики и систем управления \\
Кафедра теоретической информатики и компьютерных технологий}
\bigskip
\bigskip
\bigskip
\bigskip
\bigskip
\bigskip
\bigskip

\vfill


Лабораторная работа №4 \\
по курсу <<Моделирование>>

\bigskip

{\large <<Построение разрезов поверхностей>>}
\bigskip

\vfill



\hfill\parbox{4cm} {
Выполнил:\\
студент ИУ9-111 \hfill \\
Выборнов А. И.\hfill \medskip\\
Руководитель:\\
Домрачева А. Б.\hfill
}


\vspace{\fill}

Москва \number\year
\clearpage
}



\clearpage


\section{Постановка задачи}
    Одной из базовых задач анализа триангуляционных поверхностей является построение разрезов --- вертикальных (профилей) и горизонтальных (изолиний).

    {\it Изолиниями} уровня $h$ называют геометрическое место точек на поверхности, имеющих высоту $h$ и имеющих в любой своей окрестности другие точки с меньшей высотой:
    $I_h = \{(x,y) | z(x,y)=h, \forall \epsilon > 0 \colon \exists (x',y') \colon |(x',y'),(x,y)|<\epsilon, z(x',y')<h \}$.

    {\it Изоконтурами} между уровнями $h_1$ и $h_2$ называют замыкание геометрического места точек на поверхности, имеющих высоту $h \in [h_1, h_2)$, т.е. множество точек $I_h~=~\overline{\{ (x, y) | h_1 <= z(x,y) < h_2 \}}$.

    В задаче построения изоконтуров требуется построить множество непересекающихся регионов, каждый из которых представляет область, высоты точек внутри которой лежат в определенном диапазоне. Обычно задаётся система диапазонов с помощью начального значения самого первого диапазона, конечного значения последнего диапазона и шага построения диапазонов.

    По трёхмерной модели поверхности нужно построить множество изоконтуров, лежащих в диапазоне $[h_1, h_2)$ с шагом $\Delta h$.

\section{Теоретическая часть}
    \subsection{Построение изолиний}
        Для построения изолиний высотой h используется следующий алгоритм:

        \begin{enumerate}
            \item Помечаем каждый треугольник триангуляции, по которому проходят изолинии (т.е. выполняется условие $min( z_1, z_2, z_3 ) < h < < max( z_1, z_2, z_3 )$, где $z_i$ --- высоты трех его вершин), флагом $C_i := 1$ , а все остальные треугольники – $C_i := 0$. Если обнаружен хотя бы один треугольник, у которого хотя бы одно ребро лежит в плоскости изолинии, то $h$ уменьшается на некоторое малое $\Delta h$ и алгоритм повторяется заново.
            \item  Для каждого треугольника с $C_i = 1$ выполняем отслеживание очередной изолинии в обе стороны от данного треугольника, пока один конец не выйдет на другой или на границу триангуляции. Каждый пройденный при отслеживании треугольник помечается $C_i := 0$ . Конец алгоритма.
        \end{enumerate}

        Трудоемкость такого алгоритма, очевидно, является линейной относительно размера триангуляции.

    \subsection{Построение изоконтуров}
        Для построения множества изоконтуров, лежащих в диапазоне $[h_1, h_M)$ с шагом $\Delta h$ используется следующий алгоритм:
         \begin{enumerate}
            \item Пусть заданы уровни $h_1 ,..., h_M$, Обнуляем множества ломаных, входящих в изоконтуры: $C_i = \varnothing, i = \overline{0, M}$.
            \item Для каждого уровня $h_i$ строим изолинии. Каждую замкнутую изолинию добавляем во множество $C_i$ .
            \item Определяем все кусочки границы триангуляции между точками выхода изолиний на границу. Формируем граф, в котором в качестве узлов выступают точки выхода на границу, а в качестве рёбер ---кусочки границы между этими точками и рассчитанные изолинии. Каждая изолиния должна войти в граф дважды в виде одинаковых ориентированных рёбер, но направленных в разные стороны. Для рёбер ---кусочков границы --- устанавливаем такую ориентацию, чтобы внутренности триангуляции находились справа по ходу движения. В результате в каждом узле графа должны сходиться четыре ребра.
            \item По полученному графу строим контуры. Начинаем движение с любой вершины графа и двигаемся вперед в соответствии с ориентацией рёбер до тех пор, пока не вернемся в начальную вершину. Повторный проход по одному и тому же ребру запрещен, для чего делаются специальные пометки на рёбрах. При попадании в узел графа из граничной цепочки далее надо двигаться по ребру, cоответствующему изолинии, иначе --- по граничному ребру. Обратим внимание, что каждая изолиния войдет в два контура, соответствующих разным диапазонам высот.
            \item Для каждого полученного на предыдущем шаге контура определяем, какому диапазону высот он соответствует. Для этого нужно взять и проверить любое ребро триангуляции, входящее в составе граничной цепочки, использованной в каждом контуре. На основании этого помещаем цепочку в соответствующее множество $C_i$. Конец алгоритма.
        \end{enumerate}

        Трудоемкость данного алгоритма линейно зависит от размера триангуляции и количества изолиний.
        
\section{Реализация}
    В рамках лабораторной работы была написана программа на языке python, 

    Описание всех этапов, которые я выслал Домрачевой

\section{Тестирование}
    один результат и визуализация всех этапов.

    Результаты при разных параметрах для разных моделей



\section{Выводы}
    
    
\end{document}