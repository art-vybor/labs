% utf-8 ru, unix eolns
\documentclass[12pt,a4paper,oneside]{extarticle}
    \righthyphenmin=2 %минимально переносится 2 символа %%%
    \sloppy

% Рукопись оформлена в соответствии с правилами оформления 
% электронной версии авторского оригинала, 
% принятыми в Издательстве МГТУ им. Н.Э. Баумана.

\usepackage{geometry} % А4, примерно 28-31 строк(а) на странице 
    \geometry{paper=a4paper}
    \geometry{includehead=false} % Нет верх. колонтитула
    \geometry{includefoot=true}  % Есть номер страницы
    \geometry{bindingoffset=0mm} % Переплет    : 0  мм
    \geometry{top=20mm}          % Поле верхнее: 20 мм
    \geometry{bottom=25mm}       % Поле нижнее : 25 мм 
    \geometry{left=25mm}         % Поле левое  : 25 мм
    \geometry{right=25mm}        % Поле правое : 25 мм
    \geometry{headsep=10mm}  % От края до верх. колонтитула: 10 мм
    \geometry{footskip=20mm} % От края до нижн. колонтитула: 20 мм 

\usepackage{cmap}
\usepackage[T2A]{fontenc} 
\usepackage[utf8x]{inputenc}
\usepackage[english,russian]{babel}
\usepackage{misccorr}

\usepackage{amsmath}
\usepackage{amsfonts}
\usepackage{amssymb}

%\usepackage{cm-super} %человеческий рендер русских шрифтов

\setlength{\parindent}{1.25cm}  % Абзацный отступ: 1,25 см
\usepackage{indentfirst}        % 1-й абзац имеет отступ

\usepackage{setspace}   

\onehalfspacing % Полуторный интервал между строками

\makeatletter
\renewcommand{\@oddfoot }{\hfil\thepage\hfil} % Номер стр.
\renewcommand{\@evenfoot}{\hfil\thepage\hfil} % Номер стр.
\renewcommand{\@oddhead }{} % Нет верх. колонтитула
\renewcommand{\@evenhead}{} % Нет верх. колонтитула
\makeatother

\usepackage{fancyvrb}


\usepackage[pdftex]{graphicx}  % поддержка картинок для пдф
\graphicspath{ {./pictures/} }
\usepackage{rotating}
%\DeclareGraphicsExtensions{.jpg,.png}




\renewcommand{\labelenumi}{\theenumi.} %меняет вид нумерованного списка

\usepackage{perpage} %нумерация сносок 
\MakePerPage{footnote}

\usepackage[all]{xy} %поддержка графов

\usepackage{listings} %листинги
\renewcommand{\lstlistingname}{Листинг}
\lstset{
  basicstyle=\tiny,
  breaklines=true
  }


\usepackage{url}


\usepackage{tikz} %для рисования графиков
\usepackage{pgfplots}

\usepackage{gensymb}

\usepackage{ccaption}%изменяет подпись к рисунку
\makeatletter 
\renewcommand{\fnum@figure}[1]{Рисунок~\thefigure~---~\sffamily}
\makeatother

\begin{document}
\pgfplotsset{compat=1.8}

\thispagestyle{empty}
\newpage
{
\centering


\textbf{
МОСКОВСКИЙ ГОСУДАРСТВЕННЫЙ ТЕХНИЧЕСКИЙ УНИВЕРСИТЕТ ИМЕНИ Н. Э. БАУМАНА \\
Факультет информатики и систем управления \\
Кафедра теоретической информатики и компьютерных технологий}
\bigskip
\bigskip
\bigskip
\bigskip
\bigskip
\bigskip
\bigskip

\vfill


Лабораторная работа №5 \\
по курсу <<Моделирование>>

\bigskip

{\large <<Работа с системой моделирования GPSS>>}
\bigskip

\vfill



\hfill\parbox{4cm} {
Выполнил:\\
студент ИУ9-111 \hfill \\
Выборнов А. И.\hfill \medskip\\
Руководитель:\\
Домрачева А. Б.\hfill
}


\vspace{\fill}

Москва \number\year
\clearpage
}



\clearpage


\section{Постановка задачи}
    Кластерная система, состоящая из 25 узлов, осуществляет выполнения задач. Задачи поступают по нормальному закону распределения с матожиданием 5 секунд и дисперсией 2. Время обработки каждой задачи 5-10 минут. Кластер имеет буфер для хранения 10 задач, если все узлы заняты, то задача помещается в буфер, если буфер заполнен, то задача считается утерянной и в буфер не помещается. Провести моделирование обработки 100 задач, определить загрузку кластера и количество утерянных задач.


\section{Реализация}
    \subsection{Код программы на языке GPSS}
    \label{sec:gpss}

        \begin{lstlisting}
    }
cluster     STORAGE     25                              ; cluster of 25 nodes
            
            GENERATE    (Normal(1,5,SQR(2)))            ; generate tasks Mx=5, Dx=2
            QUEUE       buffer_queue                    ; task entered to buffer
            TEST L      Q$buffer_queue,10,unprocessed   ; if > 10 task in buffer goto unprocessed
            ENTER       cluster                         ; task sended to cluster
            DEPART      buffer_queue                    ; task leaved buffer
            ADVANCE     120,30                          ; processed task 2-3m
            LEAVE       cluster                         ; task left cluster
            TERMINATE   1                               ; task successed

unprocessed DEPART      buffer_queue                    ; task left buffer
            TERMINATE   1                               ; task not precessed

            START       100                             ; loop for 100 task
        \end{lstlisting}

    \subsection{Отчёт GPSS}
    \label{sec:report}
        Ниже представлен отчёт GPSS, полученный после выполнения программы, представленной в главе~\ref{sec:gpss}.

        \begin{lstlisting}


              GPSS World Simulation Report - Untitled Model 1.1.1


                   ???????, ???????? 19, 2015 22:23:05  

           START TIME           END TIME  BLOCKS  FACILITIES  STORAGES
                0.000            635.466    10        0          1


              NAME                       VALUE  
          BUFFER_QUEUE                10001.000
          CLUSTER                     10000.000
          UNPROCESSED                     9.000


 LABEL              LOC  BLOCK TYPE     ENTRY COUNT CURRENT COUNT RETRY
                    1    GENERATE           130             0       0
                    2    QUEUE              130             0       0
                    3    TEST               130             5       0
                    4    ENTER              125             1       0
                    5    DEPART             124             0       0
                    6    ADVANCE            124            24       0
                    7    LEAVE              100             0       0
                    8    TERMINATE          100             0       0
UNPROCESSED         9    DEPART               0             0       0
                   10    TERMINATE            0             0       0


QUEUE              MAX CONT. ENTRY ENTRY(0) AVE.CONT. AVE.TIME   AVE.(-0) RETRY
 BUFFER_QUEUE        7    6    130     73     1.407      6.879     15.689   0


STORAGE            CAP. REM. MIN. MAX.  ENTRIES AVL.  AVE.C. UTIL. RETRY DELAY
 CLUSTER            25    0   0    25      125   1   21.678  0.867    0    5
   \end{lstlisting}

\section{Тестирование}
\label{sec:test}
        \begin{figure}[h!]        
        \centering
            \begin{tikzpicture}[scale=2]
                \begin{axis}[ylabel=загрузка кластера (\%), xlabel=число узлов в кластере]
                 \tiny
                    \addplot coordinates {                            
                        (5, 97.4)
                        (10, 95.3)
                        (15, 93.5)
                        (20, 91.1)
                        (25, 86.7)
                        (30, 72.4)
                    };
                \end{axis}
            \end{tikzpicture}
        \caption{Зависимость загрузки кластера от числа узлов}
        \label{pic:test}
        \end{figure}

        \begin{figure}[h!]        
        \centering
            \begin{tikzpicture}[scale=2]
                \begin{axis}[ylabel=количество утерянных задач, xlabel=число узлов в кластере]
                 \tiny
                    \addplot coordinates {                            
                        (5, 79)
                        (10, 60)
                        (15, 33)
                        (20, 14)
                        (25, 0)
                        (30, 0)
                    };
                \end{axis}
            \end{tikzpicture}
        \caption{Зависимость количества утерянных задач от числа узлов}
        \label{pic:test2}
        \end{figure}

\section{Выводы}
    Как видно из представленного в главе~\ref{sec:report} отчёта кластер был загружен на 86.7\%, при этом не было утеряно ни одной задачи.

    Из тестирования, описанного в главе~\ref{sec:test}, видно что чем больше узлов в кластере, то тем меньше его загрузка и меньше количество утерянных задач.
    
\end{document}