% utf-8 ru, unix eolns
\documentclass[12pt,a4paper,oneside]{extarticle}
    \righthyphenmin=2 %минимально переносится 2 символа %%%
    \sloppy

% Рукопись оформлена в соответствии с правилами оформления 
% электронной версии авторского оригинала, 
% принятыми в Издательстве МГТУ им. Н.Э. Баумана.

\usepackage{geometry} % А4, примерно 28-31 строк(а) на странице 
    \geometry{paper=a4paper}
    \geometry{includehead=false} % Нет верх. колонтитула
    \geometry{includefoot=true}  % Есть номер страницы
    \geometry{bindingoffset=0mm} % Переплет    : 0  мм
    \geometry{top=20mm}          % Поле верхнее: 20 мм
    \geometry{bottom=25mm}       % Поле нижнее : 25 мм 
    \geometry{left=25mm}         % Поле левое  : 25 мм
    \geometry{right=25mm}        % Поле правое : 25 мм
    \geometry{headsep=10mm}  % От края до верх. колонтитула: 10 мм
    \geometry{footskip=20mm} % От края до нижн. колонтитула: 20 мм 

\usepackage{cmap}
\usepackage[T2A]{fontenc} 
\usepackage[utf8x]{inputenc}
\usepackage[english,russian]{babel}
\usepackage{misccorr}

\usepackage{amsmath}
\usepackage{amsfonts}
\usepackage{amssymb}

%\usepackage{cm-super} %человеческий рендер русских шрифтов

\setlength{\parindent}{1.25cm}  % Абзацный отступ: 1,25 см
\usepackage{indentfirst}        % 1-й абзац имеет отступ

\usepackage{setspace}   

\onehalfspacing % Полуторный интервал между строками

\makeatletter
\renewcommand{\@oddfoot }{\hfil\thepage\hfil} % Номер стр.
\renewcommand{\@evenfoot}{\hfil\thepage\hfil} % Номер стр.
\renewcommand{\@oddhead }{} % Нет верх. колонтитула
\renewcommand{\@evenhead}{} % Нет верх. колонтитула
\makeatother

\usepackage{fancyvrb}

\usepackage[nounderscore]{syntax} %для поддержки рбнф
%\setlength{\grammarindent}{12em} %устанавливает нужный отступ перед ::=
\setlength{\grammarparsep}{6pt plus 1pt minus 1pt}  %сокращает расстояние между правилами


\usepackage[pdftex]{graphicx}  % поддержка картинок для пдф
\graphicspath{ {./pictures/} }
\usepackage{rotating}
\usepackage{graphicx}
%\DeclareGraphicsExtensions{.jpg,.png}

\renewcommand{\labelenumi}{\theenumi.} %меняет вид нумерованного списка

\usepackage{perpage} %нумерация сносок 
\MakePerPage{footnote}

\usepackage[all]{xy} %поддержка графов

\usepackage{listings} %листинги
\renewcommand{\lstlistingname}{Листинг}
\lstset{
  basicstyle=\small,
  breaklines=true
  }

\usepackage{url}


\usepackage{tikz} %для рисования графиков
\usepackage{pgfplots}

\usepackage{rotating}

\usepackage{ccaption}%изменяет подпись к рисунку
\makeatletter 
\renewcommand{\fnum@figure}[1]{Рисунок~\thefigure~---~\sffamily}
\makeatother





\begin{document}
\pgfplotsset{compat=1.8}

\thispagestyle{empty}
\newpage
{
\centering


\textbf{
МОСКОВСКИЙ ГОСУДАРСТВЕННЫЙ ТЕХНИЧЕСКИЙ УНИВЕРСИТЕТ ИМЕНИ Н. Э. БАУМАНА \\
Факультет информатики и систем управления \\
Кафедра теоретической информатики и компьютерных технологий}
\bigskip
\bigskip
\bigskip
\bigskip
\bigskip
\bigskip
\bigskip

\vfill

\bigskip

{\large Лекции по курсу <<Математическое моделирование>>}
\bigskip

\vfill



\hfill\parbox{4cm} {
Выполнил:\\
студент ИУ9-111 \hfill \\
Выборнов А. И.\hfill \medskip\\
Лекции читала:\\
Домрачева А.Б.\hfill
}


\vspace{\fill}

Москва \number\year
\clearpage
}


\tableofcontents

\clearpage

\section{2015-09-29}

    Изучением математических моделей случайных явлений или экспериментов в первую очередь занимаются такие науки, как мат. статистика (МС) и теория вероятности (ТВ).

    Задачи МС являются обратными к задачам ТВ.
    В ТВ после задания того или иного случайного явления требуется расчитать вероятностные характеристики в рамках данной модели.
    Моделирование производится на основе результата эксперимента называемых статистическими данными.
    В ряде случаев по результатам эксперимента требуется лишь уточнить или модифицировать имеющуюся модель.
    В задачах МС вероятность того или иного события известна и необходимо оценить параметры эксперимента (параметры функции связи между двумя показателями объекта, параметры закона распределения случайной величины, в более широком случае функцию распределения случайной величины или функцию плостности распределения случайной величины).

    \subsection{Основные задачи мат. статистики}
        \begin{itemize}
            \item {\bf Задача оценки неизвестных параметров по результатам эксперимента.}
            Как правило нужно найти функцию от результатов эксперимента, является достаточно хорошей оценкой неизвестного истинного значения параметра ($a$~---~параметр, $\hat{a}$~---~оценка параметра).

            \item {\bf Задача интервального оценивания.}
            Есть строгий интервал со случайными границами (нижняя~---~$a_{\_}$, верхняя~---~$a^{\_}$), таким образом, чтобы он накрывал неизвестное истинное значение параметра с заранее заданной веростностью $\gamma$.
            $$P\{a_{-} \leq a \leq a^{-}\}=\gamma$$.

            \item  {\bf Задачи проверки статистических гипотез.}
            Требуется, на основе математических экспериментов, проверить то, или иное предположение относительно вида, и параметра функции распределения случайной величины, и функции плотности распределения случайной величины.
        \end{itemize}

\end{document}