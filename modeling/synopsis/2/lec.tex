% utf-8 ru, unix eolns
\documentclass[12pt,a4paper,oneside]{extarticle}
    \righthyphenmin=2 %минимально переносится 2 символа %%%
    \sloppy

% Рукопись оформлена в соответствии с правилами оформления 
% электронной версии авторского оригинала, 
% принятыми в Издательстве МГТУ им. Н.Э. Баумана.

\usepackage{geometry} % А4, примерно 28-31 строк(а) на странице 
    \geometry{paper=a4paper}
    \geometry{includehead=false} % Нет верх. колонтитула
    \geometry{includefoot=true}  % Есть номер страницы
    \geometry{bindingoffset=0mm} % Переплет    : 0  мм
    \geometry{top=20mm}          % Поле верхнее: 20 мм
    \geometry{bottom=25mm}       % Поле нижнее : 25 мм 
    \geometry{left=25mm}         % Поле левое  : 25 мм
    \geometry{right=25mm}        % Поле правое : 25 мм
    \geometry{headsep=10mm}  % От края до верх. колонтитула: 10 мм
    \geometry{footskip=20mm} % От края до нижн. колонтитула: 20 мм 

\usepackage{cmap}
\usepackage[T2A]{fontenc} 
\usepackage[utf8x]{inputenc}
\usepackage[english,russian]{babel}
\usepackage{misccorr}

\usepackage{amsmath}
\usepackage{amsfonts}
\usepackage{amssymb}

%\usepackage{cm-super} %человеческий рендер русских шрифтов

\setlength{\parindent}{1.25cm}  % Абзацный отступ: 1,25 см
\usepackage{indentfirst}        % 1-й абзац имеет отступ

\usepackage{setspace}   

\onehalfspacing % Полуторный интервал между строками

\makeatletter
\renewcommand{\@oddfoot }{\hfil\thepage\hfil} % Номер стр.
\renewcommand{\@evenfoot}{\hfil\thepage\hfil} % Номер стр.
\renewcommand{\@oddhead }{} % Нет верх. колонтитула
\renewcommand{\@evenhead}{} % Нет верх. колонтитула
\makeatother

\usepackage{fancyvrb}

\usepackage[nounderscore]{syntax} %для поддержки рбнф
%\setlength{\grammarindent}{12em} %устанавливает нужный отступ перед ::=
\setlength{\grammarparsep}{6pt plus 1pt minus 1pt}  %сокращает расстояние между правилами


\usepackage[pdftex]{graphicx}  % поддержка картинок для пдф
\graphicspath{ {./pictures/} }
\usepackage{rotating}
\usepackage{graphicx}
%\DeclareGraphicsExtensions{.jpg,.png}

\renewcommand{\labelenumi}{\theenumi.} %меняет вид нумерованного списка

\usepackage{perpage} %нумерация сносок 
\MakePerPage{footnote}

\usepackage[all]{xy} %поддержка графов

\usepackage{listings} %листинги
\renewcommand{\lstlistingname}{Листинг}
\lstset{
  basicstyle=\small,
  breaklines=true
  }

\usepackage{url}


\usepackage{tikz} %для рисования графиков
\usepackage{pgfplots}

\usepackage{rotating}

\usepackage{ccaption}%изменяет подпись к рисунку
\makeatletter 
\renewcommand{\fnum@figure}[1]{Рисунок~\thefigure~---~\sffamily}
\makeatother





\begin{document}
\pgfplotsset{compat=1.8}

\thispagestyle{empty}
\newpage
{
\centering


\textbf{
МОСКОВСКИЙ ГОСУДАРСТВЕННЫЙ ТЕХНИЧЕСКИЙ УНИВЕРСИТЕТ ИМЕНИ Н. Э. БАУМАНА \\
Факультет информатики и систем управления \\
Кафедра теоретической информатики и компьютерных технологий}
\bigskip
\bigskip
\bigskip
\bigskip
\bigskip
\bigskip
\bigskip

\vfill

\bigskip

{\large Лекции по курсу <<Математическое моделирование>>}
\bigskip

\vfill



\hfill\parbox{4cm} {
Выполнил:\\
студент ИУ9-111 \hfill \\
Выборнов А. И.\hfill \medskip\\
Лекции читала:\\
Домрачева А.Б.\hfill
}


\vspace{\fill}

Москва \number\year
\clearpage
}


\tableofcontents

\clearpage

\section{2015-09-29}

    Изучением математических моделей случайных явлений или экспериментов в первую очередь занимаются такие науки, как мат. статистика (МС) и теория вероятности (ТВ).

    Задачи МС являются обратными к задачам ТВ.
    В ТВ после задания того или иного случайного явления требуется расчитать вероятностные характеристики в рамках данной модели.
    Моделирование производится на основе результата эксперимента называемых статистическими данными.
    В ряде случаев по результатам эксперимента требуется лишь уточнить или модифицировать имеющуюся модель.
    В задачах МС вероятность того или иного события известна и необходимо оценить параметры эксперимента (параметры функции связи между двумя показателями объекта, параметры закона распределения случайной величины, в более широком случае функцию распределения случайной величины или функцию плостности распределения случайной величины).

    \subsection{Основные задачи мат. статистики}
        \begin{itemize}
            \item {\bf Задача оценки неизвестных параметров по результатам эксперимента.}
            Как правило нужно найти функцию от результатов эксперимента, является достаточно хорошей оценкой неизвестного истинного значения параметра ($a$~---~параметр, $\hat{a}$~---~оценка параметра).

            \item {\bf Задача интервального оценивания.}
            Есть строгий интервал со случайными границами (нижняя~---~$a_{\_}$, верхняя~---~$a^{\_}$), таким образом, чтобы он накрывал неизвестное истинное значение параметра с заранее заданной веростностью $\gamma$.
            $$P\{a_{-} \leq a \leq a^{-}\}=\gamma$$.

            \item  {\bf Задачи проверки статистических гипотез.}
            Требуется, на основе математических экспериментов, проверить то, или иное предположение относительно вида, и параметра функции распределения случайной величины, и функции плотности распределения случайной величины.
        \end{itemize}

    В мат.статистике используется выборочная терминология основанная на ''урновой'' схеме.
    Пусть имеется урна содержащая N чисел $$\{X_1, X_2, ..., X_N\}, ~(1)$$ называемая генеральной совокупностью объёмом $N$. Набор 1 может иметь бесконечную размерность.
    Из генеральной совокупности выбирается набор $$\{x_1, x_2, ..., x_n\}, n \leq N. ~(2)$$
    Набор 2 называется выборкой объёма $n$ из генеральной совокупности 1.

    Выборка может производится с возвращением и без возвращения.
    Если выборка производится с возвращением, то случайные величины в ней независимы.
    С возвращением это независимая, повторная, случайная выборка объёмом $n$. Терминология сохраняется и в случае бесконесной генеральной совокупности.

    Числа выборки 2 обычно располагают в порядке убывания или возврастания: $$\{x^{(1)}, x^{(2)}, ..., x^{(n)}\}. ~(3)$$
    Набор 3 называется вариационным рядом. Чаще всего в задачах это называется вариационный ряд.

    Эмпирической функцией распределения построенной на основе выборки 3 называется функция $\hat{F}(x) = \frac{r(x)}{n}$~($n$~---~общее число выборки, $r(x)$~---~количество элементов выборки $x_i \leq x$).

    Пример: Выборка ${0,0,9,16,21,24,29,37,42,48}$.
    
    \begin{figure}[h!]        
    \centering
        \begin{tikzpicture}[scale=1]
            \begin{axis}[] \tiny
                \addplot coordinates {                            
                        (0, 0.3)
                        (10, 0.3)
                        (10, 0.4)
                        (20, 0.4)
                        (20, 0.7)
                        (30, 0.7)
                        (30, 0.8)
                        (40, 0.8)
                        (40,1)
                        (50,1)
                };
            \end{axis}
        \end{tikzpicture}
    \caption{График $\hat{F}(x)$}
    %\label{pic:test}
    \end{figure}


    Для моделирования требуется теоретическая функция распределения случайной величины $x$. Которая может быть оценена по эмпирической функции распределения.

    По теореме гливенко-кантелли: $\sup_{x,n\rightarrow \infty}|F(x)-\hat{F_h}(x)| \rightarrow 0$

    По эмпирической функции распредления строят ... модели.

    Выборочное среднее (эмперическое среднее) - $\overline{x}=\frac{1}{n} \sum_{i=1}^{n}x_i$. Выборочный аналог первого начального момента (мат.ожидания).

    $\hat{S^2}=\frac{1}{n} \sum_{i=1}^{n}(x_i-\overline{x})$ - выборочнаяя (эмпирическая) дисперсия

    $\hat{S}=\sqrt{\frac{1}{n} \sum_{i=1}^{n}(x_i-\overline{x})}$ выборочное СКО (средне квадратичное отклонение)

    $\mu_{r,a}=(\frac{1}{n} \sum_{i=1}^{n}(x_i-a)^r)^\frac{1}{r}$ - выборочные моменты порядка $r$.

    В ряде случаев требуется оценить размах выборки $R_n=|x^{(n)} - x^{(1)}|$.

    \subsection{Точечные оценки параметров}
        Пусть имеется некоторая случайная величина $\xi$ с функцией распределения $F(x,\theta)$, плотностью распределения $f(x, \theta)$.

        Обычно говорят о параметрическом семействе распределений, в котором $\theta$ принимает различные значения.

        Вводят функцию от результатов наблюдений $$\phi=\phi(x_1,x_2,...,x_n) ~(4)$$ ($x_i$ - элемент набора 2), называемую статистикой. Задача построения точечной оценки параметра $\theta$, сводится к нахождению значения статистики. Такой что $$\hat{\theta}=\theta(x_1, x_2, ..., x_n): \sup_{n\rightarrow \infty} |\hat{\theta} - \theta| \rightarrow 0$$

        Необходимо установить эффективную оценку, рекомендуемую в качестве результата.

        \subsubsection{Свойства оценок}
        $\hat{\theta}=\theta(x_1, x_2, ..., x_n)$

        Пример:
        $\hat{\lambda}=1/\overline{x} = \frac{1}{\frac{1}{n} \sum_{i=1}^{n}x_i} = \frac{n}{\sum_{i=1}^{n}x_i} = \frac{n}{x_1+x_2+...+x_n}$

        $\hat{\lambda}=\lambda(x_1, x_2, ..., x_n)=\frac{n}{x_1+x_2+...+x_n}$

        $M\hat{\theta}=\theta$

        Оценка $\hat{\theta}$ является несмещённой оценкой параметра $\theta$, если её мат. ожидание совпадает с теоритической величиной.

        $\lim_{n \rightarrow \infty} M\hat{\theta_n}=\theta$

        Таким образом на свойство оценок влияет объём выборки.

        Если $\lim_{n \rightarrow \infty} P\{|\hat{\theta_n}-\theta| < \epsilon \}\rightarrow 1$ мы говорим о состоятельности оценок. Сходится по вероятности.

        Пусть $\hat{\theta_n}$ асимптотически несмещённая оценка параметра $\theta$. В случае когда $\lim_{n\rightarrow\infty} S^2(\hat{\theta_n)}\rightarrow 0$ оценка состоятельна.

        Таким образом асимптотическая несмещённость оценки $\theta$ и минимизация разброса значений параметра при $n \rightarrow \infty$ обеспечивают состоятельность оценки~(теорема приводится без доказательства).

        Пусть имеются две оценки $\hat{\theta_n}^1$, $\hat{\theta_n}^2$.

        $S^2(\hat{\theta_n}^1)=M(\hat{\theta_n}^1 - \theta)^2 \leq M(\hat{\theta_n}^2 - \theta)^2 = S^2(\hat{\theta_n}^2)$

        То оценка $S^2(\hat{\theta_n}^1)$ является более эффективной по сравнению с $S^2(\hat{\theta_n}^2)$.

\section{2015-10-06}
    \subsection{Доверительные интервалы}
        Пусть $x = \{x_1, x_2, ... x_n\} \in X_N, F(x, \theta), f(x, \theta)$.
        Пусть для параметра $\theta$ построен интервал $\theta \in \[ \theta_-, \theta^- \] \colon$
        $$\theta_-=\theta_-(x_1, x_2, ..., x_n),~\theta^-=\theta^-(x_1, x_2, ..., x_n).$$
        Этот интервал называется доверительным для параметра $\theta$ с коэффициентом доверия $\gamma \colon P\{\theta_- <= \theta <= \theta^-\}=\gamma$.

        \subsubsection{Методы построения доверительных интервалов}
            Статистикой метода называется любая функция $T=T(x_1, x_2, ..., x_n)$, если $T=T(x_1, x_2, ..., x_n, \theta)$, то говорят о центральной статистике. При этом выборка $\{x_1, x_2, ..., x_n\}$ - независимая случайная повторная из генеральной совокупности $X_N$, параметр $\theta$ скалярная величина, но в общем случае может рассматриваться как вектор. Функция $T$ является монотонной относительно параметра $\theta$.

            Квантилем уровня $\alpha$ функции распределения $F(x, \theta)$.
            


\end{document}