
% utf-8 ru, unix eolns
\documentclass[12pt,a4paper,oneside]{extarticle}
    \righthyphenmin=2 %минимально переносится 2 символа %%%
    \sloppy

% Рукопись оформлена в соответствии с правилами оформления 
% электронной версии авторского оригинала, 
% принятыми в Издательстве МГТУ им. Н.Э. Баумана.

\usepackage{geometry} % А4, примерно 28-31 строк(а) на странице 
    \geometry{paper=a4paper}
    \geometry{includehead=false} % Нет верх. колонтитула
    \geometry{includefoot=true}  % Есть номер страницы
    \geometry{bindingoffset=0mm} % Переплет    : 0  мм
    \geometry{top=20mm}          % Поле верхнее: 20 мм
    \geometry{bottom=25mm}       % Поле нижнее : 25 мм 
    \geometry{left=25mm}         % Поле левое  : 25 мм
    \geometry{right=25mm}        % Поле правое : 25 мм
    \geometry{headsep=10mm}  % От края до верх. колонтитула: 10 мм
    \geometry{footskip=20mm} % От края до нижн. колонтитула: 20 мм 

\usepackage{cmap}
\usepackage[T2A]{fontenc} 
\usepackage[utf8x]{inputenc}
\usepackage[english,russian]{babel}
\usepackage{misccorr}

\usepackage{amsmath}
\usepackage{amsfonts}
\usepackage{amssymb}

%\usepackage{cm-super} %человеческий рендер русских шрифтов

\setlength{\parindent}{1.25cm}  % Абзацный отступ: 1,25 см
\usepackage{indentfirst}        % 1-й абзац имеет отступ

\usepackage{setspace}   

\onehalfspacing % Полуторный интервал между строками

\makeatletter
\renewcommand{\@oddfoot }{\hfil\thepage\hfil} % Номер стр.
\renewcommand{\@evenfoot}{\hfil\thepage\hfil} % Номер стр.
\renewcommand{\@oddhead }{} % Нет верх. колонтитула
\renewcommand{\@evenhead}{} % Нет верх. колонтитула
\makeatother

\usepackage{fancyvrb}


\usepackage[pdftex]{graphicx}  % поддержка картинок для пдф
\graphicspath{ {./pictures/} }
\usepackage{rotating}
%\DeclareGraphicsExtensions{.jpg,.png}




\renewcommand{\labelenumi}{\theenumi.} %меняет вид нумерованного списка

\usepackage{perpage} %нумерация сносок 
\MakePerPage{footnote}

\usepackage[all]{xy} %поддержка графов

\usepackage{listings} %листинги
\renewcommand{\lstlistingname}{Листинг}
\lstset{
  basicstyle=\tiny,
  breaklines=true
  }


\usepackage{url}


\usepackage{tikz} %для рисования графиков
\usepackage{pgfplots}

\usepackage{gensymb}

\usepackage{ccaption}%изменяет подпись к рисунку
\makeatletter 
\renewcommand{\fnum@figure}[1]{Рисунок~\thefigure~---~\sffamily}
\makeatother

\begin{document}
\pgfplotsset{compat=1.8}

\thispagestyle{empty}
\newpage
{
\centering


\textbf{
МОСКОВСКИЙ ГОСУДАРСТВЕННЫЙ ТЕХНИЧЕСКИЙ УНИВЕРСИТЕТ ИМЕНИ Н. Э. БАУМАНА \\
Факультет информатики и систем управления \\
Кафедра теоретической информатики и компьютерных технологий}
\bigskip
\bigskip
\bigskip
\bigskip
\bigskip
\bigskip
\bigskip

\vfill


Лабораторная работа №1 \\
по курсу <<Современные вычислительные методы>>

\bigskip

{\large <<Численное решение одномерной краевой задачи (глобальные базисные функции)>>}
\bigskip

\vfill



\hfill\parbox{4cm} {
Выполнил:\\
студент ИУ9-111 \hfill \\
Выборнов А. И.\hfill \medskip\\
Руководитель:\\
Басараб М.А.\hfill
}


\vspace{\fill}

Москва \number\year
\clearpage
}

\clearpage

% Отчет должен содержать:
% - постановка исходной задачи (вариант);
% - задача с однородными краевыми условиями;
% - выражения для расчета коэффициентов СЛАУ;
% - графики результатов расчетов (приближенное решение);
% - графики зависимости погрешности от числа базисных функций;
% - исходный код программы.


\section{Постановка задачи}
    Найти минимум:
    \begin{equation*}
        \begin{cases}
            f_1(x) = 2x_1^2 + x_1x_2 + x_2^2 - 6x_1 - 5x_2 \rightarrow \min, \\
            f_1(x) = 3(x_1 - 2)^2 + 2(x_2 - 5)^2 \rightarrow \min, \\
            g_1(x) = x_1 + x_2 \le 6, \\
            g_2(x) = 3x_2 - 2x_1 \le 10.
        \end{cases}
    \end{equation*}

\section{Задача с однородными краевыми условиями}
\section{Задача с однородными краевыми условиями}
\section{Выражения для расчётов коэффициентов СЛАУ}

\section{Анализ результата}
    % графики результатов для разных k
    % график зависимости погрешности от числа базисных функций

\section{Исходный код программы на языке программирования python}
    \lstinputlisting{../doit.py}

\section{Результат работы}

\end{document}